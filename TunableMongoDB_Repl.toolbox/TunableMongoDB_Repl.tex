\batchmode %% Suppresses most terminal output.
\documentclass{article}
\usepackage{color}
\definecolor{boxshade}{gray}{0.85}
\setlength{\textwidth}{360pt}
\setlength{\textheight}{541pt}
\usepackage{latexsym}
\usepackage{ifthen}
% \usepackage{color}
%%%%%%%%%%%%%%%%%%%%%%%%%%%%%%%%%%%%%%%%%%%%%%%%%%%%%%%%%%%%%%%%%%%%%%%%%%%%%
% SWITCHES                                                                  %
%%%%%%%%%%%%%%%%%%%%%%%%%%%%%%%%%%%%%%%%%%%%%%%%%%%%%%%%%%%%%%%%%%%%%%%%%%%%%
\newboolean{shading} 
\setboolean{shading}{false}
\makeatletter
 %% this is needed only when inserted into the file, not when
 %% used as a package file.
%%%%%%%%%%%%%%%%%%%%%%%%%%%%%%%%%%%%%%%%%%%%%%%%%%%%%%%%%%%%%%%%%%%%%%%%%%%%%
%                                                                           %
% DEFINITIONS OF SYMBOL-PRODUCING COMMANDS                                  %
%                                                                           %
%    TLA+      LaTeX                                                        %
%    symbol    command                                                      %
%    ------    -------                                                      %
%    =>        \implies                                                     %
%    <:        \ltcolon                                                     %
%    :>        \colongt                                                     %
%    ==        \defeq                                                       %
%    ..        \dotdot                                                      %
%    ::        \coloncolon                                                  %
%    =|        \eqdash                                                      %
%    ++        \pp                                                          %
%    --        \mm                                                          %
%    **        \stst                                                        %
%    //        \slsl                                                        %
%    ^         \ct                                                          %
%    \A        \A                                                           %
%    \E        \E                                                           %
%    \AA       \AA                                                          %
%    \EE       \EE                                                          %
%%%%%%%%%%%%%%%%%%%%%%%%%%%%%%%%%%%%%%%%%%%%%%%%%%%%%%%%%%%%%%%%%%%%%%%%%%%%%
\newlength{\symlength}
\newcommand{\implies}{\Rightarrow}
\newcommand{\ltcolon}{\mathrel{<\!\!\mbox{:}}}
\newcommand{\colongt}{\mathrel{\!\mbox{:}\!\!>}}
\newcommand{\defeq}{\;\mathrel{\smash   %% keep this symbol from being too tall
    {{\stackrel{\scriptscriptstyle\Delta}{=}}}}\;}
\newcommand{\dotdot}{\mathrel{\ldotp\ldotp}}
\newcommand{\coloncolon}{\mathrel{::\;}}
\newcommand{\eqdash}{\mathrel = \joinrel \hspace{-.28em}|}
\newcommand{\pp}{\mathbin{++}}
\newcommand{\mm}{\mathbin{--}}
\newcommand{\stst}{*\!*}
\newcommand{\slsl}{/\!/}
\newcommand{\ct}{\hat{\hspace{.4em}}}
\newcommand{\A}{\forall}
\newcommand{\E}{\exists}
\renewcommand{\AA}{\makebox{$\raisebox{.05em}{\makebox[0pt][l]{%
   $\forall\hspace{-.517em}\forall\hspace{-.517em}\forall$}}%
   \forall\hspace{-.517em}\forall \hspace{-.517em}\forall\,$}}
\newcommand{\EE}{\makebox{$\raisebox{.05em}{\makebox[0pt][l]{%
   $\exists\hspace{-.517em}\exists\hspace{-.517em}\exists$}}%
   \exists\hspace{-.517em}\exists\hspace{-.517em}\exists\,$}}
\newcommand{\whileop}{\.{\stackrel
  {\mbox{\raisebox{-.3em}[0pt][0pt]{$\scriptscriptstyle+\;\,$}}}%
  {-\hspace{-.16em}\triangleright}}}

% Commands are defined to produce the upper-case keywords.
% Note that some have space after them.
\newcommand{\ASSUME}{\textsc{assume }}
\newcommand{\ASSUMPTION}{\textsc{assumption }}
\newcommand{\AXIOM}{\textsc{axiom }}
\newcommand{\BOOLEAN}{\textsc{boolean }}
\newcommand{\CASE}{\textsc{case }}
\newcommand{\CONSTANT}{\textsc{constant }}
\newcommand{\CONSTANTS}{\textsc{constants }}
\newcommand{\ELSE}{\settowidth{\symlength}{\THEN}%
   \makebox[\symlength][l]{\textsc{ else}}}
\newcommand{\EXCEPT}{\textsc{ except }}
\newcommand{\EXTENDS}{\textsc{extends }}
\newcommand{\FALSE}{\textsc{false}}
\newcommand{\IF}{\textsc{if }}
\newcommand{\IN}{\settowidth{\symlength}{\LET}%
   \makebox[\symlength][l]{\textsc{in}}}
\newcommand{\INSTANCE}{\textsc{instance }}
\newcommand{\LET}{\textsc{let }}
\newcommand{\LOCAL}{\textsc{local }}
\newcommand{\MODULE}{\textsc{module }}
\newcommand{\OTHER}{\textsc{other }}
\newcommand{\STRING}{\textsc{string}}
\newcommand{\THEN}{\textsc{ then }}
\newcommand{\THEOREM}{\textsc{theorem }}
\newcommand{\LEMMA}{\textsc{lemma }}
\newcommand{\PROPOSITION}{\textsc{proposition }}
\newcommand{\COROLLARY}{\textsc{corollary }}
\newcommand{\TRUE}{\textsc{true}}
\newcommand{\VARIABLE}{\textsc{variable }}
\newcommand{\VARIABLES}{\textsc{variables }}
\newcommand{\WITH}{\textsc{ with }}
\newcommand{\WF}{\textrm{WF}}
\newcommand{\SF}{\textrm{SF}}
\newcommand{\CHOOSE}{\textsc{choose }}
\newcommand{\ENABLED}{\textsc{enabled }}
\newcommand{\UNCHANGED}{\textsc{unchanged }}
\newcommand{\SUBSET}{\textsc{subset }}
\newcommand{\UNION}{\textsc{union }}
\newcommand{\DOMAIN}{\textsc{domain }}
% Added for tla2tex
\newcommand{\BY}{\textsc{by }}
\newcommand{\OBVIOUS}{\textsc{obvious }}
\newcommand{\HAVE}{\textsc{have }}
\newcommand{\QED}{\textsc{qed }}
\newcommand{\TAKE}{\textsc{take }}
\newcommand{\DEF}{\textsc{ def }}
\newcommand{\HIDE}{\textsc{hide }}
\newcommand{\RECURSIVE}{\textsc{recursive }}
\newcommand{\USE}{\textsc{use }}
\newcommand{\DEFINE}{\textsc{define }}
\newcommand{\PROOF}{\textsc{proof }}
\newcommand{\WITNESS}{\textsc{witness }}
\newcommand{\PICK}{\textsc{pick }}
\newcommand{\DEFS}{\textsc{defs }}
\newcommand{\PROVE}{\settowidth{\symlength}{\ASSUME}%
   \makebox[\symlength][l]{\textsc{prove}}\@s{-4.1}}%
  %% The \@s{-4.1) is a kludge added on 24 Oct 2009 [happy birthday, Ellen]
  %% so the correct alignment occurs if the user types
  %%   ASSUME X
  %%   PROVE  Y
  %% because it cancels the extra 4.1 pts added because of the 
  %% extra space after the PROVE.  This seems to works OK.
  %% However, the 4.1 equals Parameters.LaTeXLeftSpace(1) and
  %% should be changed if that method ever changes.
\newcommand{\SUFFICES}{\textsc{suffices }}
\newcommand{\NEW}{\textsc{new }}
\newcommand{\LAMBDA}{\textsc{lambda }}
\newcommand{\STATE}{\textsc{state }}
\newcommand{\ACTION}{\textsc{action }}
\newcommand{\TEMPORAL}{\textsc{temporal }}
\newcommand{\ONLY}{\textsc{only }}              %% added by LL on 2 Oct 2009
\newcommand{\OMITTED}{\textsc{omitted }}        %% added by LL on 31 Oct 2009
\newcommand{\@pfstepnum}[2]{\ensuremath{\langle#1\rangle}\textrm{#2}}
\newcommand{\bang}{\@s{1}\mbox{\small !}\@s{1}}
%% We should format || differently in PlusCal code than in TLA+ formulas.
\newcommand{\p@barbar}{\ifpcalsymbols
   \,\,\rule[-.25em]{.075em}{1em}\hspace*{.2em}\rule[-.25em]{.075em}{1em}\,\,%
   \else \,||\,\fi}
%% PlusCal keywords
\newcommand{\p@fair}{\textbf{fair }}
\newcommand{\p@semicolon}{\textbf{\,; }}
\newcommand{\p@algorithm}{\textbf{algorithm }}
\newcommand{\p@mmfair}{\textbf{-{}-fair }}
\newcommand{\p@mmalgorithm}{\textbf{-{}-algorithm }}
\newcommand{\p@assert}{\textbf{assert }}
\newcommand{\p@await}{\textbf{await }}
\newcommand{\p@begin}{\textbf{begin }}
\newcommand{\p@end}{\textbf{end }}
\newcommand{\p@call}{\textbf{call }}
\newcommand{\p@define}{\textbf{define }}
\newcommand{\p@do}{\textbf{ do }}
\newcommand{\p@either}{\textbf{either }}
\newcommand{\p@or}{\textbf{or }}
\newcommand{\p@goto}{\textbf{goto }}
\newcommand{\p@if}{\textbf{if }}
\newcommand{\p@then}{\,\,\textbf{then }}
\newcommand{\p@else}{\ifcsyntax \textbf{else } \else \,\,\textbf{else }\fi}
\newcommand{\p@elsif}{\,\,\textbf{elsif }}
\newcommand{\p@macro}{\textbf{macro }}
\newcommand{\p@print}{\textbf{print }}
\newcommand{\p@procedure}{\textbf{procedure }}
\newcommand{\p@process}{\textbf{process }}
\newcommand{\p@return}{\textbf{return}}
\newcommand{\p@skip}{\textbf{skip}}
\newcommand{\p@variable}{\textbf{variable }}
\newcommand{\p@variables}{\textbf{variables }}
\newcommand{\p@while}{\textbf{while }}
\newcommand{\p@when}{\textbf{when }}
\newcommand{\p@with}{\textbf{with }}
\newcommand{\p@lparen}{\textbf{(\,\,}}
\newcommand{\p@rparen}{\textbf{\,\,) }}   
\newcommand{\p@lbrace}{\textbf{\{\,\,}}   
\newcommand{\p@rbrace}{\textbf{\,\,\} }}

%%%%%%%%%%%%%%%%%%%%%%%%%%%%%%%%%%%%%%%%%%%%%%%%%%%%%%%%%
% REDEFINE STANDARD COMMANDS TO MAKE THEM FORMAT BETTER %
%                                                       %
% We redefine \in and \notin                            %
%%%%%%%%%%%%%%%%%%%%%%%%%%%%%%%%%%%%%%%%%%%%%%%%%%%%%%%%%
\renewcommand{\_}{\rule{.4em}{.06em}\hspace{.05em}}
\newlength{\equalswidth}
\let\oldin=\in
\let\oldnotin=\notin
\renewcommand{\in}{%
   {\settowidth{\equalswidth}{$\.{=}$}\makebox[\equalswidth][c]{$\oldin$}}}
\renewcommand{\notin}{%
   {\settowidth{\equalswidth}{$\.{=}$}\makebox[\equalswidth]{$\oldnotin$}}}


%%%%%%%%%%%%%%%%%%%%%%%%%%%%%%%%%%%%%%%%%%%%%%%%%%%%
%                                                  %
% HORIZONTAL BARS:                                 %
%                                                  %
%   \moduleLeftDash    |~~~~~~~~~~                 %
%   \moduleRightDash    ~~~~~~~~~~|                %
%   \midbar            |----------|                %
%   \bottombar         |__________|                %
%%%%%%%%%%%%%%%%%%%%%%%%%%%%%%%%%%%%%%%%%%%%%%%%%%%%
\newlength{\charwidth}\settowidth{\charwidth}{{\small\tt M}}
\newlength{\boxrulewd}\setlength{\boxrulewd}{.4pt}
\newlength{\boxlineht}\setlength{\boxlineht}{.5\baselineskip}
\newcommand{\boxsep}{\charwidth}
\newlength{\boxruleht}\setlength{\boxruleht}{.5ex}
\newlength{\boxruledp}\setlength{\boxruledp}{-\boxruleht}
\addtolength{\boxruledp}{\boxrulewd}
\newcommand{\boxrule}{\leaders\hrule height \boxruleht depth \boxruledp
                      \hfill\mbox{}}
\newcommand{\@computerule}{%
  \setlength{\boxruleht}{.5ex}%
  \setlength{\boxruledp}{-\boxruleht}%
  \addtolength{\boxruledp}{\boxrulewd}}

\newcommand{\bottombar}{\hspace{-\boxsep}%
  \raisebox{-\boxrulewd}[0pt][0pt]{\rule[.5ex]{\boxrulewd}{\boxlineht}}%
  \boxrule
  \raisebox{-\boxrulewd}[0pt][0pt]{%
      \rule[.5ex]{\boxrulewd}{\boxlineht}}\hspace{-\boxsep}\vspace{0pt}}

\newcommand{\moduleLeftDash}%
   {\hspace*{-\boxsep}%
     \raisebox{-\boxlineht}[0pt][0pt]{\rule[.5ex]{\boxrulewd
               }{\boxlineht}}%
    \boxrule\hspace*{.4em }}

\newcommand{\moduleRightDash}%
    {\hspace*{.4em}\boxrule
    \raisebox{-\boxlineht}[0pt][0pt]{\rule[.5ex]{\boxrulewd
               }{\boxlineht}}\hspace{-\boxsep}}%\vspace{.2em}

\newcommand{\midbar}{\hspace{-\boxsep}\raisebox{-.5\boxlineht}[0pt][0pt]{%
   \rule[.5ex]{\boxrulewd}{\boxlineht}}\boxrule\raisebox{-.5\boxlineht%
   }[0pt][0pt]{\rule[.5ex]{\boxrulewd}{\boxlineht}}\hspace{-\boxsep}}

%%%%%%%%%%%%%%%%%%%%%%%%%%%%%%%%%%%%%%%%%%%%%%%%%%%%%%%%%%%%%%%%%%%%%%%%%%%%%
% FORMATING COMMANDS                                                        %
%%%%%%%%%%%%%%%%%%%%%%%%%%%%%%%%%%%%%%%%%%%%%%%%%%%%%%%%%%%%%%%%%%%%%%%%%%%%%

%%%%%%%%%%%%%%%%%%%%%%%%%%%%%%%%%%%%%%%%%%%%%%%%%%%%%%%%%%%%%%%%%%%%%%%%%%%%%
% PLUSCAL SHADING                                                           %
%%%%%%%%%%%%%%%%%%%%%%%%%%%%%%%%%%%%%%%%%%%%%%%%%%%%%%%%%%%%%%%%%%%%%%%%%%%%%

% The TeX pcalshading switch is set on to cause PlusCal shading to be
% performed.  This changes the behavior of the following commands and
% environments to cause full-width shading to be performed on all lines.
% 
%   \tstrut \@x cpar mcom \@pvspace
% 
% The TeX pcalsymbols switch is turned on when typesetting a PlusCal algorithm,
% whether or not shading is being performed.  It causes symbols (other than
% parentheses and braces and PlusCal-only keywords) that should be typeset
% differently depending on whether they are in an algorithm to be typeset
% appropriately.  Currently, the only such symbol is "||".
%
% The TeX csyntax switch is turned on when typesetting a PlusCal algorithm in
% c-syntax.  This allows symbols to be format differently in the two syntaxes.
% The "else" keyword is the only one that is.

\newif\ifpcalshading \pcalshadingfalse
\newif\ifpcalsymbols \pcalsymbolsfalse
\newif\ifcsyntax     \csyntaxtrue

% The \@pvspace command makes a vertical space.  It uses \vspace
% except with \ifpcalshading, in which case it sets \pvcalvspace
% and the space is added by a following \@x command.
%
\newlength{\pcalvspace}\setlength{\pcalvspace}{0pt}%
\newcommand{\@pvspace}[1]{%
  \ifpcalshading
     \par\global\setlength{\pcalvspace}{#1}%
  \else
     \par\vspace{#1}%
  \fi
}

% The lcom environment was changed to set \lcomindent equal to
% the indentation it produces.  This length is used by the
% cpar environment to make shading extend for the full width
% of the line.  This assumes that lcom environments are not
% nested.  I hope TLATeX does not nest them.
%
\newlength{\lcomindent}%
\setlength{\lcomindent}{0pt}%

%\tstrut: A strut to produce inter-paragraph space in a comment.
%\rstrut: A strut to extend the bottom of a one-line comment so
%         there's no break in the shading between comments on 
%         successive lines.
\newcommand\tstrut%
  {\raisebox{\vshadelen}{\raisebox{-.25em}{\rule{0pt}{1.15em}}}%
   \global\setlength{\vshadelen}{0pt}}
\newcommand\rstrut{\raisebox{-.25em}{\rule{0pt}{1.15em}}%
 \global\setlength{\vshadelen}{0pt}}


% \.{op} formats operator op in math mode with empty boxes on either side.
% Used because TeX otherwise vary the amount of space it leaves around op.
\renewcommand{\.}[1]{\ensuremath{\mbox{}#1\mbox{}}}

% \@s{n} produces an n-point space
\newcommand{\@s}[1]{\hspace{#1pt}}           

% \@x{txt} starts a specification line in the beginning with txt
% in the final LaTeX source.
\newlength{\@xlen}
\newcommand\xtstrut%
  {\setlength{\@xlen}{1.05em}%
   \addtolength{\@xlen}{\pcalvspace}%
    \raisebox{\vshadelen}{\raisebox{-.25em}{\rule{0pt}{\@xlen}}}%
   \global\setlength{\vshadelen}{0pt}%
   \global\setlength{\pcalvspace}{0pt}}

\newcommand{\@x}[1]{\par
  \ifpcalshading
  \makebox[0pt][l]{\shadebox{\xtstrut\hspace*{\textwidth}}}%
  \fi
  \mbox{$\mbox{}#1\mbox{}$}}  

% \@xx{txt} continues a specification line with the text txt.
\newcommand{\@xx}[1]{\mbox{$\mbox{}#1\mbox{}$}}  

% \@y{cmt} produces a one-line comment.
\newcommand{\@y}[1]{\mbox{\footnotesize\hspace{.65em}%
  \ifthenelse{\boolean{shading}}{%
      \shadebox{#1\hspace{-\the\lastskip}\rstrut}}%
               {#1\hspace{-\the\lastskip}\rstrut}}}

% \@z{cmt} produces a zero-width one-line comment.
\newcommand{\@z}[1]{\makebox[0pt][l]{\footnotesize
  \ifthenelse{\boolean{shading}}{%
      \shadebox{#1\hspace{-\the\lastskip}\rstrut}}%
               {#1\hspace{-\the\lastskip}\rstrut}}}


% \@w{str} produces the TLA+ string "str".
\newcommand{\@w}[1]{\textsf{``{#1}''}}             


%%%%%%%%%%%%%%%%%%%%%%%%%%%%%%%%%%%%%%%%%%%%%%%%%%%%%%%%%%%%%%%%%%%%%%%%%%%%%
% SHADING                                                                   %
%%%%%%%%%%%%%%%%%%%%%%%%%%%%%%%%%%%%%%%%%%%%%%%%%%%%%%%%%%%%%%%%%%%%%%%%%%%%%
\def\graymargin{1}
  % The number of points of margin in the shaded box.

% \definecolor{boxshade}{gray}{.85}
% Defines the darkness of the shading: 1 = white, 0 = black
% Added by TLATeX only if needed.

% \shadebox{txt} puts txt in a shaded box.
\newlength{\templena}
\newlength{\templenb}
\newsavebox{\tempboxa}
\newcommand{\shadebox}[1]{{\setlength{\fboxsep}{\graymargin pt}%
     \savebox{\tempboxa}{#1}%
     \settoheight{\templena}{\usebox{\tempboxa}}%
     \settodepth{\templenb}{\usebox{\tempboxa}}%
     \hspace*{-\fboxsep}\raisebox{0pt}[\templena][\templenb]%
        {\colorbox{boxshade}{\usebox{\tempboxa}}}\hspace*{-\fboxsep}}}

% \vshade{n} makes an n-point inter-paragraph space, with
%  shading if the `shading' flag is true.
\newlength{\vshadelen}
\setlength{\vshadelen}{0pt}
\newcommand{\vshade}[1]{\ifthenelse{\boolean{shading}}%
   {\global\setlength{\vshadelen}{#1pt}}%
   {\vspace{#1pt}}}

\newlength{\boxwidth}
\newlength{\multicommentdepth}

%%%%%%%%%%%%%%%%%%%%%%%%%%%%%%%%%%%%%%%%%%%%%%%%%%%%%%%%%%%%%%%%%%%%%%%%%%%%%
% THE cpar ENVIRONMENT                                                      %
% ^^^^^^^^^^^^^^^^^^^^                                                      %
% The LaTeX input                                                           %
%                                                                           %
%   \begin{cpar}{pop}{nest}{isLabel}{d}{e}{arg6}                            %
%     XXXXXXXXXXXXXXX                                                       %
%     XXXXXXXXXXXXXXX                                                       %
%     XXXXXXXXXXXXXXX                                                       %
%   \end{cpar}                                                              %
%                                                                           %
% produces one of two possible results.  If isLabel is the letter "T",      %
% it produces the following, where [label] is the result of typesetting     %
% arg6 in an LR box, and d is is a number representing a distance in        %
% points.                                                                   %
%                                                                           %
%   prevailing |<-- d -->[label]<- e ->XXXXXXXXXXXXXXX                      %
%         left |                       XXXXXXXXXXXXXXX                      %
%       margin |                       XXXXXXXXXXXXXXX                      %
%                                                                           %
% If isLabel is the letter "F", then it produces                            %
%                                                                           %
%   prevailing |<-- d -->XXXXXXXXXXXXXXXXXXXXXXX                            %
%         left |         <- e ->XXXXXXXXXXXXXXXX                            %
%       margin |                XXXXXXXXXXXXXXXX                            %
%                                                                           %
% where d and e are numbers representing distances in points.               %
%                                                                           %
% The prevailing left margin is the one in effect before the most recent    %
% pop (argument 1) cpar environments with "T" as the nest argument, where   %
% pop is a number \geq 0.                                                   %
%                                                                           %
% If the nest argument is the letter "T", then the prevailing left          %
% margin is moved to the left of the second (and following) lines of        %
% X's.  Otherwise, the prevailing left margin is left unchanged.            %
%                                                                           %
% An \unnest{n} command moves the prevailing left margin to where it was    %
% before the most recent n cpar environments with "T" as the nesting        %
% argument.                                                                 %
%                                                                           %
% The environment leaves no vertical space above or below it, or between    %
% its paragraphs.  (TLATeX inserts the proper amount of vertical space.)    %
%%%%%%%%%%%%%%%%%%%%%%%%%%%%%%%%%%%%%%%%%%%%%%%%%%%%%%%%%%%%%%%%%%%%%%%%%%%%%

\newcounter{pardepth}
\setcounter{pardepth}{0}

% \setgmargin{txt} defines \gmarginN to be txt, where N is \roman{pardepth}.
% \thegmargin equals \gmarginN, where N is \roman{pardepth}.
\newcommand{\setgmargin}[1]{%
  \expandafter\xdef\csname gmargin\roman{pardepth}\endcsname{#1}}
\newcommand{\thegmargin}{\csname gmargin\roman{pardepth}\endcsname}
\newcommand{\gmargin}{0pt}

\newsavebox{\tempsbox}

\newlength{\@cparht}
\newlength{\@cpardp}
\newenvironment{cpar}[6]{%
  \addtocounter{pardepth}{-#1}%
  \ifthenelse{\boolean{shading}}{\par\begin{lrbox}{\tempsbox}%
                                 \begin{minipage}[t]{\linewidth}}{}%
  \begin{list}{}{%
     \edef\temp{\thegmargin}
     \ifthenelse{\equal{#3}{T}}%
       {\settowidth{\leftmargin}{\hspace{\temp}\footnotesize #6\hspace{#5pt}}%
        \addtolength{\leftmargin}{#4pt}}%
       {\setlength{\leftmargin}{#4pt}%
        \addtolength{\leftmargin}{#5pt}%
        \addtolength{\leftmargin}{\temp}%
        \setlength{\itemindent}{-#5pt}}%
      \ifthenelse{\equal{#2}{T}}{\addtocounter{pardepth}{1}%
                                 \setgmargin{\the\leftmargin}}{}%
      \setlength{\labelwidth}{0pt}%
      \setlength{\labelsep}{0pt}%
      \setlength{\itemindent}{-\leftmargin}%
      \setlength{\topsep}{0pt}%
      \setlength{\parsep}{0pt}%
      \setlength{\partopsep}{0pt}%
      \setlength{\parskip}{0pt}%
      \setlength{\itemsep}{0pt}
      \setlength{\itemindent}{#4pt}%
      \addtolength{\itemindent}{-\leftmargin}}%
   \ifthenelse{\equal{#3}{T}}%
      {\item[\tstrut\footnotesize \hspace{\temp}{#6}\hspace{#5pt}]
        }%
      {\item[\tstrut\hspace{\temp}]%
         }%
   \footnotesize}
 {\hspace{-\the\lastskip}\tstrut
 \end{list}%
  \ifthenelse{\boolean{shading}}%
          {\end{minipage}%
           \end{lrbox}%
           \ifpcalshading
             \setlength{\@cparht}{\ht\tempsbox}%
             \setlength{\@cpardp}{\dp\tempsbox}%
             \addtolength{\@cparht}{.15em}%
             \addtolength{\@cpardp}{.2em}%
             \addtolength{\@cparht}{\@cpardp}%
            % I don't know what's going on here.  I want to add a
            % \pcalvspace high shaded line, but I don't know how to
            % do it.  A little trial and error shows that the following
            % does a reasonable job approximating that, eliminating
            % the line if \pcalvspace is small.
            \addtolength{\@cparht}{\pcalvspace}%
             \ifdim \pcalvspace > .8em
               \addtolength{\pcalvspace}{-.2em}%
               \hspace*{-\lcomindent}%
               \shadebox{\rule{0pt}{\pcalvspace}\hspace*{\textwidth}}\par
               \global\setlength{\pcalvspace}{0pt}%
               \fi
             \hspace*{-\lcomindent}%
             \makebox[0pt][l]{\raisebox{-\@cpardp}[0pt][0pt]{%
                 \shadebox{\rule{0pt}{\@cparht}\hspace*{\textwidth}}}}%
             \hspace*{\lcomindent}\usebox{\tempsbox}%
             \par
           \else
             \shadebox{\usebox{\tempsbox}}\par
           \fi}%
           {}%
  }

%%%%%%%%%%%%%%%%%%%%%%%%%%%%%%%%%%%%%%%%%%%%%%%%%%%%%%%%%%%%%%%%%%%%%%%%%%%%%%
% THE ppar ENVIRONMENT                                                       %
% ^^^^^^^^^^^^^^^^^^^^                                                       %
% The environment                                                            %
%                                                                            %
%   \begin{ppar} ... \end{ppar}                                              %
%                                                                            %
% is equivalent to                                                           %
%                                                                            %
%   \begin{cpar}{0}{F}{F}{0}{0}{} ... \end{cpar}                             %
%                                                                            %
% The environment is put around each line of the output for a PlusCal        %
% algorithm.                                                                 %
%%%%%%%%%%%%%%%%%%%%%%%%%%%%%%%%%%%%%%%%%%%%%%%%%%%%%%%%%%%%%%%%%%%%%%%%%%%%%%
%\newenvironment{ppar}{%
%  \ifthenelse{\boolean{shading}}{\par\begin{lrbox}{\tempsbox}%
%                                 \begin{minipage}[t]{\linewidth}}{}%
%  \begin{list}{}{%
%     \edef\temp{\thegmargin}
%        \setlength{\leftmargin}{0pt}%
%        \addtolength{\leftmargin}{\temp}%
%        \setlength{\itemindent}{0pt}%
%      \setlength{\labelwidth}{0pt}%
%      \setlength{\labelsep}{0pt}%
%      \setlength{\itemindent}{-\leftmargin}%
%      \setlength{\topsep}{0pt}%
%      \setlength{\parsep}{0pt}%
%      \setlength{\partopsep}{0pt}%
%      \setlength{\parskip}{0pt}%
%      \setlength{\itemsep}{0pt}
%      \setlength{\itemindent}{0pt}%
%      \addtolength{\itemindent}{-\leftmargin}}%
%      \item[\tstrut\hspace{\temp}]}%
% {\hspace{-\the\lastskip}\tstrut
% \end{list}%
%  \ifthenelse{\boolean{shading}}{\end{minipage}  
%                                 \end{lrbox}%
%                                 \shadebox{\usebox{\tempsbox}}\par}{}%
%  }

 %%% TESTING
 \newcommand{\xtest}[1]{\par
 \makebox[0pt][l]{\shadebox{\xtstrut\hspace*{\textwidth}}}%
 \mbox{$\mbox{}#1\mbox{}$}} 

% \newcommand{\xxtest}[1]{\par
% \makebox[0pt][l]{\shadebox{\xtstrut{#1}\hspace*{\textwidth}}}%
% \mbox{$\mbox{}#1\mbox{}$}} 

%\newlength{\pcalvspace}
%\setlength{\pcalvspace}{0pt}
% \newlength{\xxtestlen}
% \setlength{\xxtestlen}{0pt}
% \newcommand\xtstrut%
%   {\setlength{\xxtestlen}{1.15em}%
%    \addtolength{\xxtestlen}{\pcalvspace}%
%     \raisebox{\vshadelen}{\raisebox{-.25em}{\rule{0pt}{\xxtestlen}}}%
%    \global\setlength{\vshadelen}{0pt}%
%    \global\setlength{\pcalvspace}{0pt}}
   
   %%%% TESTING
   
   %% The xcpar environment
   %%  Note: overloaded use of \pcalvspace for testing.
   %%
%   \newlength{\xcparht}%
%   \newlength{\xcpardp}%
   
%   \newenvironment{xcpar}[6]{%
%  \addtocounter{pardepth}{-#1}%
%  \ifthenelse{\boolean{shading}}{\par\begin{lrbox}{\tempsbox}%
%                                 \begin{minipage}[t]{\linewidth}}{}%
%  \begin{list}{}{%
%     \edef\temp{\thegmargin}%
%     \ifthenelse{\equal{#3}{T}}%
%       {\settowidth{\leftmargin}{\hspace{\temp}\footnotesize #6\hspace{#5pt}}%
%        \addtolength{\leftmargin}{#4pt}}%
%       {\setlength{\leftmargin}{#4pt}%
%        \addtolength{\leftmargin}{#5pt}%
%        \addtolength{\leftmargin}{\temp}%
%        \setlength{\itemindent}{-#5pt}}%
%      \ifthenelse{\equal{#2}{T}}{\addtocounter{pardepth}{1}%
%                                 \setgmargin{\the\leftmargin}}{}%
%      \setlength{\labelwidth}{0pt}%
%      \setlength{\labelsep}{0pt}%
%      \setlength{\itemindent}{-\leftmargin}%
%      \setlength{\topsep}{0pt}%
%      \setlength{\parsep}{0pt}%
%      \setlength{\partopsep}{0pt}%
%      \setlength{\parskip}{0pt}%
%      \setlength{\itemsep}{0pt}%
%      \setlength{\itemindent}{#4pt}%
%      \addtolength{\itemindent}{-\leftmargin}}%
%   \ifthenelse{\equal{#3}{T}}%
%      {\item[\xtstrut\footnotesize \hspace{\temp}{#6}\hspace{#5pt}]%
%        }%
%      {\item[\xtstrut\hspace{\temp}]%
%         }%
%   \footnotesize}
% {\hspace{-\the\lastskip}\tstrut
% \end{list}%
%  \ifthenelse{\boolean{shading}}{\end{minipage}  
%                                 \end{lrbox}%
%                                 \setlength{\xcparht}{\ht\tempsbox}%
%                                 \setlength{\xcpardp}{\dp\tempsbox}%
%                                 \addtolength{\xcparht}{.15em}%
%                                 \addtolength{\xcpardp}{.2em}%
%                                 \addtolength{\xcparht}{\xcpardp}%
%                                 \hspace*{-\lcomindent}%
%                                 \makebox[0pt][l]{\raisebox{-\xcpardp}[0pt][0pt]{%
%                                      \shadebox{\rule{0pt}{\xcparht}\hspace*{\textwidth}}}}%
%                                 \hspace*{\lcomindent}\usebox{\tempsbox}%
%                                 \par}{}%
%  }
%  
% \newlength{\xmcomlen}
%\newenvironment{xmcom}[1]{%
%  \setcounter{pardepth}{0}%
%  \hspace{.65em}%
%  \begin{lrbox}{\alignbox}\sloppypar%
%      \setboolean{shading}{false}%
%      \setlength{\boxwidth}{#1pt}%
%      \addtolength{\boxwidth}{-.65em}%
%      \begin{minipage}[t]{\boxwidth}\footnotesize
%      \parskip=0pt\relax}%
%       {\end{minipage}\end{lrbox}%
%       \setlength{\xmcomlen}{\textwidth}%
%       \addtolength{\xmcomlen}{-\wd\alignbox}%
%       \settodepth{\alignwidth}{\usebox{\alignbox}}%
%       \global\setlength{\multicommentdepth}{\alignwidth}%
%       \setlength{\boxwidth}{\alignwidth}%
%       \global\addtolength{\alignwidth}{-\maxdepth}%
%       \addtolength{\boxwidth}{.1em}%
%       \raisebox{0pt}[0pt][0pt]{%
%        \ifthenelse{\boolean{shading}}%
%          {\hspace*{-\xmcomlen}\shadebox{\rule[-\boxwidth]{0pt}{0pt}%
%                                 \hspace*{\xmcomlen}\usebox{\alignbox}}}%
%          {\usebox{\alignbox}}}%
%       \vspace*{\alignwidth}\pagebreak[0]\vspace{-\alignwidth}\par}
% % a multi-line comment, whose first argument is its width in points.
%  
   
%%%%%%%%%%%%%%%%%%%%%%%%%%%%%%%%%%%%%%%%%%%%%%%%%%%%%%%%%%%%%%%%%%%%%%%%%%%%%%
% THE lcom ENVIRONMENT                                                       %
% ^^^^^^^^^^^^^^^^^^^^                                                       %
% A multi-line comment with no text to its left is typeset in an lcom        % 
% environment, whose argument is a number representing the indentation       % 
% of the left margin, in points.  All the text of the comment should be      % 
% inside cpar environments.                                                  % 
%%%%%%%%%%%%%%%%%%%%%%%%%%%%%%%%%%%%%%%%%%%%%%%%%%%%%%%%%%%%%%%%%%%%%%%%%%%%%%
\newenvironment{lcom}[1]{%
  \setlength{\lcomindent}{#1pt} % Added for PlusCal handling.
  \par\vspace{.2em}%
  \sloppypar
  \setcounter{pardepth}{0}%
  \footnotesize
  \begin{list}{}{%
    \setlength{\leftmargin}{#1pt}
    \setlength{\labelwidth}{0pt}%
    \setlength{\labelsep}{0pt}%
    \setlength{\itemindent}{0pt}%
    \setlength{\topsep}{0pt}%
    \setlength{\parsep}{0pt}%
    \setlength{\partopsep}{0pt}%
    \setlength{\parskip}{0pt}}
    \item[]}%
  {\end{list}\vspace{.3em}\setlength{\lcomindent}{0pt}%
 }


%%%%%%%%%%%%%%%%%%%%%%%%%%%%%%%%%%%%%%%%%%%%%%%%%%%%%%%%%%%%%%%%%%%%%%%%%%%%%
% THE mcom ENVIRONMENT AND \mutivspace COMMAND                              %
% ^^^^^^^^^^^^^^^^^^^^^^^^^^^^^^^^^^^^^^^^^^^^                              %
%                                                                           %
% A part of the spec containing a right-comment of the form                 %
%                                                                           %
%      xxxx (*************)                                                 %
%      yyyy (* ccccccccc *)                                                 %
%      ...  (* ccccccccc *)                                                 %
%           (* ccccccccc *)                                                 %
%           (* ccccccccc *)                                                 %
%           (*************)                                                 %
%                                                                           %
% is typeset by                                                             %
%                                                                           %
%     XXXX \begin{mcom}{d}                                                  %
%            CCCC ... CCC                                                   %
%          \end{mcom}                                                       %
%     YYYY ...                                                              %
%     \multivspace{n}                                                       %
%                                                                           %
% where the number d is the width in points of the comment, n is the        %
% number of xxxx, yyyy, ...  lines to the left of the comment.              %
% All the text of the comment should be typeset in cpar environments.       %
%                                                                           %
% This puts the comment into a single box (so no page breaks can occur      %
% within it).  The entire box is shaded iff the shading flag is true.       %
%%%%%%%%%%%%%%%%%%%%%%%%%%%%%%%%%%%%%%%%%%%%%%%%%%%%%%%%%%%%%%%%%%%%%%%%%%%%%
\newlength{\xmcomlen}%
\newenvironment{mcom}[1]{%
  \setcounter{pardepth}{0}%
  \hspace{.65em}%
  \begin{lrbox}{\alignbox}\sloppypar%
      \setboolean{shading}{false}%
      \setlength{\boxwidth}{#1pt}%
      \addtolength{\boxwidth}{-.65em}%
      \begin{minipage}[t]{\boxwidth}\footnotesize
      \parskip=0pt\relax}%
       {\end{minipage}\end{lrbox}%
       \setlength{\xmcomlen}{\textwidth}%       % For PlusCal shading
       \addtolength{\xmcomlen}{-\wd\alignbox}%  % For PlusCal shading
       \settodepth{\alignwidth}{\usebox{\alignbox}}%
       \global\setlength{\multicommentdepth}{\alignwidth}%
       \setlength{\boxwidth}{\alignwidth}%      % For PlusCal shading
       \global\addtolength{\alignwidth}{-\maxdepth}%
       \addtolength{\boxwidth}{.1em}%           % For PlusCal shading
      \raisebox{0pt}[0pt][0pt]{%
        \ifthenelse{\boolean{shading}}%
          {\ifpcalshading
             \hspace*{-\xmcomlen}%
             \shadebox{\rule[-\boxwidth]{0pt}{0pt}\hspace*{\xmcomlen}%
                          \usebox{\alignbox}}%
           \else
             \shadebox{\usebox{\alignbox}}
           \fi
          }%
          {\usebox{\alignbox}}}%
       \vspace*{\alignwidth}\pagebreak[0]\vspace{-\alignwidth}\par}
 % a multi-line comment, whose first argument is its width in points.


% \multispace{n} produces the vertical space indicated by "|"s in 
% this situation
%   
%     xxxx (*************)
%     xxxx (* ccccccccc *)
%      |   (* ccccccccc *)
%      |   (* ccccccccc *)
%      |   (* ccccccccc *)
%      |   (*************)
%
% where n is the number of "xxxx" lines.
\newcommand{\multivspace}[1]{\addtolength{\multicommentdepth}{-#1\baselineskip}%
 \addtolength{\multicommentdepth}{1.2em}%
 \ifthenelse{\lengthtest{\multicommentdepth > 0pt}}%
    {\par\vspace{\multicommentdepth}\par}{}}

%\newenvironment{hpar}[2]{%
%  \begin{list}{}{\setlength{\leftmargin}{#1pt}%
%                 \addtolength{\leftmargin}{#2pt}%
%                 \setlength{\itemindent}{-#2pt}%
%                 \setlength{\topsep}{0pt}%
%                 \setlength{\parsep}{0pt}%
%                 \setlength{\partopsep}{0pt}%
%                 \setlength{\parskip}{0pt}%
%                 \addtolength{\labelsep}{0pt}}%
%  \item[]\footnotesize}{\end{list}}
%    %%%%%%%%%%%%%%%%%%%%%%%%%%%%%%%%%%%%%%%%%%%%%%%%%%%%%%%%%%%%%%%%%%%%%%%%
%    % Typesets a sequence of paragraphs like this:                         %
%    %                                                                      %
%    %      left |<-- d1 --> XXXXXXXXXXXXXXXXXXXXXXXX                       %
%    %    margin |           <- d2 -> XXXXXXXXXXXXXXX                       %
%    %           |                    XXXXXXXXXXXXXXX                       %
%    %           |                                                          %
%    %           |                    XXXXXXXXXXXXXXX                       %
%    %           |                    XXXXXXXXXXXXXXX                       %
%    %                                                                      %
%    % where d1 = #1pt and d2 = #2pt, but with no vspace between            %
%    % paragraphs.                                                          %
%    %%%%%%%%%%%%%%%%%%%%%%%%%%%%%%%%%%%%%%%%%%%%%%%%%%%%%%%%%%%%%%%%%%%%%%%%

%%%%%%%%%%%%%%%%%%%%%%%%%%%%%%%%%%%%%%%%%%%%%%%%%%%%%%%%%%%%%%%%%%%%%%
% Commands for repeated characters that produce dashes.              %
%%%%%%%%%%%%%%%%%%%%%%%%%%%%%%%%%%%%%%%%%%%%%%%%%%%%%%%%%%%%%%%%%%%%%%
% \raisedDash{wd}{ht}{thk} makes a horizontal line wd characters wide, 
% raised a distance ht ex's above the baseline, with a thickness of 
% thk em's.
\newcommand{\raisedDash}[3]{\raisebox{#2ex}{\setlength{\alignwidth}{.5em}%
  \rule{#1\alignwidth}{#3em}}}

% The following commands take a single argument n and produce the
% output for n repeated characters, as follows
%   \cdash:    -
%   \tdash:    ~
%   \ceqdash:  =
%   \usdash:   _
\newcommand{\cdash}[1]{\raisedDash{#1}{.5}{.04}}
\newcommand{\usdash}[1]{\raisedDash{#1}{0}{.04}}
\newcommand{\ceqdash}[1]{\raisedDash{#1}{.5}{.08}}
\newcommand{\tdash}[1]{\raisedDash{#1}{1}{.08}}

\newlength{\spacewidth}
\setlength{\spacewidth}{.2em}
\newcommand{\e}[1]{\hspace{#1\spacewidth}}
%% \e{i} produces space corresponding to i input spaces.


%% Alignment-file Commands

\newlength{\alignboxwidth}
\newlength{\alignwidth}
\newsavebox{\alignbox}

% \al{i}{j}{txt} is used in the alignment file to put "%{i}{j}{wd}"
% in the log file, where wd is the width of the line up to that point,
% and txt is the following text.
\newcommand{\al}[3]{%
  \typeout{\%{#1}{#2}{\the\alignwidth}}%
  \cl{#3}}

%% \cl{txt} continues a specification line in the alignment file
%% with text txt.
\newcommand{\cl}[1]{%
  \savebox{\alignbox}{\mbox{$\mbox{}#1\mbox{}$}}%
  \settowidth{\alignboxwidth}{\usebox{\alignbox}}%
  \addtolength{\alignwidth}{\alignboxwidth}%
  \usebox{\alignbox}}

% \fl{txt} in the alignment file begins a specification line that
% starts with the text txt.
\newcommand{\fl}[1]{%
  \par
  \savebox{\alignbox}{\mbox{$\mbox{}#1\mbox{}$}}%
  \settowidth{\alignwidth}{\usebox{\alignbox}}%
  \usebox{\alignbox}}



  
%%%%%%%%%%%%%%%%%%%%%%%%%%%%%%%%%%%%%%%%%%%%%%%%%%%%%%%%%%%%%%%%%%%%%%%%%%%%%
% Ordinarily, TeX typesets letters in math mode in a special math italic    %
% font.  This makes it typeset "it" to look like the product of the         %
% variables i and t, rather than like the word "it".  The following         %
% commands tell TeX to use an ordinary italic font instead.                 %
%%%%%%%%%%%%%%%%%%%%%%%%%%%%%%%%%%%%%%%%%%%%%%%%%%%%%%%%%%%%%%%%%%%%%%%%%%%%%
\ifx\documentclass\undefined
\else
  \DeclareSymbolFont{tlaitalics}{\encodingdefault}{cmr}{m}{it}
  \let\itfam\symtlaitalics
\fi

\makeatletter
\newcommand{\tlx@c}{\c@tlx@ctr\advance\c@tlx@ctr\@ne}
\newcounter{tlx@ctr}
\c@tlx@ctr=\itfam \multiply\c@tlx@ctr"100\relax \advance\c@tlx@ctr "7061\relax
\mathcode`a=\tlx@c \mathcode`b=\tlx@c \mathcode`c=\tlx@c \mathcode`d=\tlx@c
\mathcode`e=\tlx@c \mathcode`f=\tlx@c \mathcode`g=\tlx@c \mathcode`h=\tlx@c
\mathcode`i=\tlx@c \mathcode`j=\tlx@c \mathcode`k=\tlx@c \mathcode`l=\tlx@c
\mathcode`m=\tlx@c \mathcode`n=\tlx@c \mathcode`o=\tlx@c \mathcode`p=\tlx@c
\mathcode`q=\tlx@c \mathcode`r=\tlx@c \mathcode`s=\tlx@c \mathcode`t=\tlx@c
\mathcode`u=\tlx@c \mathcode`v=\tlx@c \mathcode`w=\tlx@c \mathcode`x=\tlx@c
\mathcode`y=\tlx@c \mathcode`z=\tlx@c
\c@tlx@ctr=\itfam \multiply\c@tlx@ctr"100\relax \advance\c@tlx@ctr "7041\relax
\mathcode`A=\tlx@c \mathcode`B=\tlx@c \mathcode`C=\tlx@c \mathcode`D=\tlx@c
\mathcode`E=\tlx@c \mathcode`F=\tlx@c \mathcode`G=\tlx@c \mathcode`H=\tlx@c
\mathcode`I=\tlx@c \mathcode`J=\tlx@c \mathcode`K=\tlx@c \mathcode`L=\tlx@c
\mathcode`M=\tlx@c \mathcode`N=\tlx@c \mathcode`O=\tlx@c \mathcode`P=\tlx@c
\mathcode`Q=\tlx@c \mathcode`R=\tlx@c \mathcode`S=\tlx@c \mathcode`T=\tlx@c
\mathcode`U=\tlx@c \mathcode`V=\tlx@c \mathcode`W=\tlx@c \mathcode`X=\tlx@c
\mathcode`Y=\tlx@c \mathcode`Z=\tlx@c
\makeatother

%%%%%%%%%%%%%%%%%%%%%%%%%%%%%%%%%%%%%%%%%%%%%%%%%%%%%%%%%%
%                THE describe ENVIRONMENT                %
%%%%%%%%%%%%%%%%%%%%%%%%%%%%%%%%%%%%%%%%%%%%%%%%%%%%%%%%%%
%
%
% It is like the description environment except it takes an argument
% ARG that should be the text of the widest label.  It adjusts the
% indentation so each item with label LABEL produces
%%      LABEL             blah blah blah
%%      <- width of ARG ->blah blah blah
%%                        blah blah blah
\newenvironment{describe}[1]%
   {\begin{list}{}{\settowidth{\labelwidth}{#1}%
            \setlength{\labelsep}{.5em}%
            \setlength{\leftmargin}{\labelwidth}% 
            \addtolength{\leftmargin}{\labelsep}%
            \addtolength{\leftmargin}{\parindent}%
            \def\makelabel##1{\rm ##1\hfill}}%
            \setlength{\topsep}{0pt}}%% 
                % Sets \topsep to 0 to reduce vertical space above
                % and below embedded displayed equations
   {\end{list}}

%   For tlatex.TeX
\usepackage{verbatim}
\makeatletter
\def\tla{\let\%\relax%
         \@bsphack
         \typeout{\%{\the\linewidth}}%
             \let\do\@makeother\dospecials\catcode`\^^M\active
             \let\verbatim@startline\relax
             \let\verbatim@addtoline\@gobble
             \let\verbatim@processline\relax
             \let\verbatim@finish\relax
             \verbatim@}
\let\endtla=\@esphack

\let\pcal=\tla
\let\endpcal=\endtla
\let\ppcal=\tla
\let\endppcal=\endtla

% The tlatex environment is used by TLATeX.TeX to typeset TLA+.
% TLATeX.TLA starts its files by writing a \tlatex command.  This
% command/environment sets \parindent to 0 and defines \% to its
% standard definition because the writing of the log files is messed up
% if \% is defined to be something else.  It also executes
% \@computerule to determine the dimensions for the TLA horizonatl
% bars.
\newenvironment{tlatex}{\@computerule%
                        \setlength{\parindent}{0pt}%
                       \makeatletter\chardef\%=`\%}{}


% The notla environment produces no output.  You can turn a 
% tla environment to a notla environment to prevent tlatex.TeX from
% re-formatting the environment.

\def\notla{\let\%\relax%
         \@bsphack
             \let\do\@makeother\dospecials\catcode`\^^M\active
             \let\verbatim@startline\relax
             \let\verbatim@addtoline\@gobble
             \let\verbatim@processline\relax
             \let\verbatim@finish\relax
             \verbatim@}
\let\endnotla=\@esphack

\let\nopcal=\notla
\let\endnopcal=\endnotla
\let\noppcal=\notla
\let\endnoppcal=\endnotla

%%%%%%%%%%%%%%%%%%%%%%%% end of tlatex.sty file %%%%%%%%%%%%%%%%%%%%%%% 
% last modified on Fri  3 August 2012 at 14:23:49 PST by lamport

\begin{document}
\tlatex
\setboolean{shading}{true}
 \@x{}\moduleLeftDash\@xx{ {\MODULE}
 TunableMongoDB\_Repl}\moduleRightDash\@xx{}%
\@x{ {\EXTENDS} Naturals ,\, FiniteSets ,\, Sequences ,\, TLC}%
\@pvspace{8.0pt}%
\@x{}%
\@y{\@s{0}%
 constants and variables
}%
\@xx{}%
\@x{ {\CONSTANTS} Client ,\, Server ,\,\@s{8.2}}%
\@y{\@s{0}%
 the set of clients and servers
}%
\@xx{}%
\@x{\@s{54.75} Key ,\, Value ,\,\@s{20.5}}%
\@y{\@s{0}%
 the set of keys and values
}%
\@xx{}%
\@x{\@s{54.75} Nil ,\,\@s{52.61}}%
\@y{\@s{0}%
 model value, place holder
}%
\@xx{}%
\@x{\@s{54.75} PtStop\@s{42.91}}%
\@y{\@s{0}%
 max physical time
}%
\@xx{}%
\@pvspace{8.0pt}%
\@x{ {\VARIABLES} Primary ,\,\@s{32.72}}%
\@y{\@s{0}%
 Primary node
}%
\@xx{}%
\@x{\@s{51.42} Secondary ,\,\@s{25.32}}%
\@y{\@s{0}%
 secondary nodes
}%
\@xx{}%
\@x{\@s{51.42} Oplog ,\,\@s{44.62}}%
\@y{\@s{0}%
 \ensuremath{oplog[s]}: \ensuremath{oplog} at \ensuremath{server[s]
}}%
\@xx{}%
\@x{\@s{51.42} Store ,\,\@s{46.92}}%
\@y{\@s{0}%
 \ensuremath{store[s]}: data stored at \ensuremath{server[s]
}}%
\@xx{}%
\@x{\@s{51.42} Ct ,\,\@s{58.61}}%
\@y{\@s{0}%
 \ensuremath{Ct[s]}: cluster time at node \ensuremath{s
}}%
\@xx{}%
\@x{\@s{51.42} Ot ,\,\@s{58.10}}%
\@y{\@s{0}%
 \ensuremath{Ot[s]}: the last applied operation time at server \ensuremath{s
}}%
\@xx{}%
\@x{\@s{51.42} ServerMsg ,\,\@s{23.64}}%
\@y{\@s{0}%
 \ensuremath{ServerMsg[s]}: the channel of heartbeat \ensuremath{msgs} at
 server \ensuremath{s
}}%
\@xx{}%
\@x{\@s{51.42} Pt ,\,\@s{58.98}}%
\@y{\@s{0}%
 \ensuremath{Pt[s]}: physical time at server \ensuremath{s
}}%
\@xx{}%
\@x{\@s{51.42} Cp ,\,\@s{57.14}}%
\@y{\@s{0}%
 \ensuremath{Cp[s]}: majority commit point at server \ensuremath{s
}}%
\@xx{}%
\@x{\@s{51.42} State ,\,\@s{47.31}}%
\@y{\@s{0}%
 \ensuremath{State[s]}: the latest \ensuremath{Ot} of all servers that server
 \ensuremath{s} knows
}%
\@xx{}%
\@x{\@s{51.42} CalState ,\,\@s{32.48}}%
\@y{\@s{0}%
 \ensuremath{CalState}: sorted \ensuremath{State[Primary]
}}%
\@xx{}%
\@x{\@s{51.42} CurrentTerm ,\,\@s{11.90}}%
\@y{\@s{0}%
 \ensuremath{CurrentTerm[s]}: current election term at server \ensuremath{s
}}%
\@xx{}%
\@x{\@s{127.57}}%
\@y{\@s{0}%
 \ensuremath{\.{\rightarrow}} updated in \ensuremath{update\_position},
 heartbeat and replicate
}%
\@xx{}%
\@x{\@s{51.42} ReadyToServe ,\,\@s{8.2}}%
\@y{\@s{0}%
 equal to 0 before any primary is elected
}%
\@xx{}%
\@x{\@s{51.42} SyncSource\@s{26.20}}%
\@y{\@s{0}%
 \ensuremath{SyncSource[s]}: sync source of server node \ensuremath{s
}}%
\@xx{}%
\@pvspace{8.0pt}%
\@x{}\midbar\@xx{}%
\@x{ {\ASSUME} Cardinality ( Client )\@s{1.66} \.{\geq} 1\@s{4.10}}%
\@y{\@s{0}%
 at least one clinet
}%
\@xx{}%
\@x{ {\ASSUME} Cardinality ( Server ) \.{\geq} 2\@s{4.10}}%
\@y{\@s{0}%
 at least one primary and one secondary
}%
\@xx{}%
\@x{ {\ASSUME} Cardinality ( Key ) \.{\geq} 1\@s{4.1}}%
\@y{\@s{0}%
 at least one object
}%
\@xx{}%
\@x{ {\ASSUME} Cardinality ( Value ) \.{\geq} 2\@s{4.1}}%
\@y{\@s{0}%
 at least two values to update
}%
\@xx{}%
\@pvspace{8.0pt}%
\@x{}%
\@y{\@s{0}%
 Helpers
}%
\@xx{}%
\@x{}\midbar\@xx{}%
\@x{ HLCLt ( x ,\, y ) \.{\defeq} {\IF} x . p \.{<} y . p}%
\@x{\@s{79.86} \.{\THEN} {\TRUE}}%
\@x{\@s{75.76} \.{\ELSE} {\IF} x . p \.{=} y . p}%
\@x{\@s{79.86} \.{\THEN} {\IF} x . l \.{<} y . l}%
\@x{\@s{123.33} \.{\THEN} {\TRUE}}%
\@x{\@s{111.17} \.{\ELSE} {\FALSE}}%
\@x{\@s{79.86} \.{\ELSE} {\FALSE}}%
\@pvspace{8.0pt}%
 \@x{ HLCMin ( x ,\, y )\@s{1.49} \.{\defeq} {\IF} HLCLt ( x ,\, y ) \.{\THEN}
 x \.{\ELSE} y}%
 \@x{ HLCMax ( x ,\, y ) \.{\defeq} {\IF} HLCLt ( x ,\, y ) \.{\THEN}
 y\@s{0.10} \.{\ELSE} x}%
\@x{ HLCType \.{\defeq} [ p \.{:} Nat ,\, l \.{:} Nat ]}%
 \@x{ Min ( x ,\, y )\@s{1.49} \.{\defeq} {\IF} x \.{<} y \.{\THEN} x
 \.{\ELSE} y}%
\@x{ Max ( x ,\, y ) \.{\defeq} {\IF} x \.{>} y \.{\THEN} x \.{\ELSE} y}%
\@pvspace{8.0pt}%
 \@x{ vars \.{\defeq} {\langle} Primary ,\, Secondary ,\, Oplog ,\, Store ,\,
 Ct ,\, Ot ,\, ServerMsg ,\,}%
\@x{\@s{41.61} Pt ,\, Cp ,\, CalState ,\, State ,\,}%
\@x{\@s{41.61} CurrentTerm ,\, ReadyToServe ,\, SyncSource {\rangle}}%
\@pvspace{8.0pt}%
\@x{ {\RECURSIVE} CreateState ( \_ ,\, \_ )}%
\@y{\@s{0}%
 init state
}%
\@xx{}%
\@x{ CreateState ( len ,\, seq ) \.{\defeq}}%
\@x{\@s{16.4} {\IF} len \.{=} 0 \.{\THEN} seq}%
 \@x{\@s{16.4} \.{\ELSE} CreateState ( len \.{-} 1 ,\, Append ( seq ,\, [ p
 \.{\mapsto} 0 ,\, l \.{\mapsto} 0 ] ) )}%
\@pvspace{8.0pt}%
 \@x{ LogTerm ( i ,\, index ) \.{\defeq} {\IF} index \.{=} 0 \.{\THEN} 0
 \.{\ELSE} Oplog [ i ] [ index ] . term}%
\@x{ LastTerm ( i ) \.{\defeq} CurrentTerm [ i ]}%
\@pvspace{8.0pt}%
\@x{}%
\@y{\@s{0}%
 Is node \ensuremath{i} ahead of node \ensuremath{j
}}%
\@xx{}%
 \@x{ NotBehind ( i ,\, j ) \.{\defeq} \.{\lor} LastTerm ( i ) \.{>} LastTerm
 ( j )}%
\@x{\@s{88.82} \.{\lor} \.{\land} LastTerm ( i ) \.{=} LastTerm ( j )}%
\@x{\@s{99.93} \.{\land} Len ( Oplog [ i ] ) \.{\geq} Len ( Oplog [ j ] )}%
\@pvspace{8.0pt}%
 \@x{ IsMajority ( servers ) \.{\defeq} Cardinality ( servers ) \.{*} 2 \.{>}
 Cardinality ( Server )}%
\@pvspace{8.0pt}%
\@x{}%
\@y{\@s{0}%
 Return the maximum value from a set, or undefined if the set is empty.
}%
\@xx{}%
 \@x{ MaxVal ( s ) \.{\defeq} {\CHOOSE} x \.{\in} s \.{:} \A\, y \.{\in} s
 \.{:} x \.{\geq} y}%
\@pvspace{8.0pt}%
\@x{}%
\@y{\@s{0}%
 commit point
}%
\@xx{}%
\@x{ {\RECURSIVE} AddState ( \_ ,\, \_ ,\, \_ )}%
\@x{ AddState ( new ,\, state ,\, index ) \.{\defeq}}%
\@x{\@s{16.4} {\IF} index \.{=} 1 \.{\land} HLCLt ( new ,\, state [ 1 ] )}%
\@x{\@s{32.65} \.{\THEN}\@s{4.1} {\langle} new {\rangle} \.{\circ} state}%
\@y{\@s{0}%
 less than the first
}%
\@xx{}%
\@x{\@s{16.4} \.{\ELSE} {\IF} index \.{=} Len ( state ) \.{+} 1}%
\@x{\@s{32.8} \.{\THEN} state \.{\circ} {\langle} new {\rangle}}%
\@x{\@s{16.4} \.{\ELSE} {\IF} HLCLt ( new ,\, state [ index ] )}%
 \@x{\@s{32.8} \.{\THEN} SubSeq ( state ,\, 1 ,\, index\@s{5.63} \.{-} 1 )
 \.{\circ} {\langle} new {\rangle} \.{\circ} SubSeq ( state ,\, index ,\, Len
 ( state ) )}%
\@x{\@s{16.4} \.{\ELSE} AddState ( new ,\, state ,\, index \.{+} 1 )}%
\@pvspace{8.0pt}%
\@x{ {\RECURSIVE} RemoveState ( \_ ,\, \_ ,\, \_ )}%
\@x{ RemoveState ( old ,\, state ,\, index ) \.{\defeq}}%
\@x{\@s{16.4} {\IF} state [ index ] \.{=} old}%
 \@x{\@s{32.65} \.{\THEN} SubSeq ( state ,\, 1 ,\, index \.{-} 1 ) \.{\circ}
 SubSeq ( state ,\, index \.{+} 1 ,\, Len ( state ) )}%
\@x{\@s{16.4} \.{\ELSE} RemoveState ( old ,\, state ,\, index \.{+} 1 )}%
\@pvspace{8.0pt}%
 \@x{ AdvanceState ( new ,\, old ,\, state ) \.{\defeq} AddState ( new ,\,
 RemoveState ( old ,\, state ,\, 1 ) ,\, 1 )}%
\@pvspace{8.0pt}%
\@x{}%
\@y{\@s{0}%
 clock
}%
\@xx{}%
\@pvspace{8.0pt}%
 \@x{ MaxPt \.{\defeq} \.{\LET} x \.{\defeq} {\CHOOSE} s \.{\in} Server \.{:}
 \A\, s1 \.{\in} Server \.{\,\backslash\,} \{ s \} \.{:}}%
\@x{\@s{141.48} Pt [ s ] \.{\geq} Pt [ s1 ] \.{\IN} Pt [ x ]}%
\@pvspace{8.0pt}%
 \@x{ Tick ( s ) \.{\defeq} Ct \.{'} \.{=} {\IF} Ct [ s ] . p \.{\geq} Pt [ s
 ]}%
 \@x{\@s{91.79} \.{\THEN} [ Ct {\EXCEPT} {\bang} [ s ] \.{=} [ p \.{\mapsto} @
 . p ,\, l \.{\mapsto} @ . l \.{+} 1 ] ]}%
 \@x{\@s{79.63} \.{\ELSE} [ Ct {\EXCEPT} {\bang} [ s ] \.{=} [ p \.{\mapsto}
 Pt [ s ] ,\, l \.{\mapsto} 0 ] ]}%
\@pvspace{8.0pt}%
\@x{}%
\@y{\@s{0}%
 heartbeat
}%
\@xx{}%
\@x{}%
\@y{\@s{0}%
 Only \ensuremath{Primary} node sends heartbeat once advance pt
}%
\@xx{}%
\@x{ BroadcastHeartbeat ( s ) \.{\defeq}}%
 \@x{\@s{16.4} \.{\LET} msg \.{\defeq} [ type \.{\mapsto}\@w{heartbeat} ,\, s
 \.{\mapsto} s ,\, aot \.{\mapsto} Ot [ s ] ,\,}%
 \@x{\@s{76.21} ct \.{\mapsto} Ct [ s ] ,\, cp\@s{11.23} \.{\mapsto} Cp [ s ]
 ,\, term \.{\mapsto} CurrentTerm [ s ] ]}%
 \@x{\@s{16.4} \.{\IN} ServerMsg \.{'} \.{=} [ x \.{\in} Server \.{\mapsto}
 {\IF} x \.{=} s \.{\THEN} ServerMsg [ x ]}%
\@x{\@s{202.01} \.{\ELSE} Append ( ServerMsg [ x ] ,\, msg ) ]}%
\@pvspace{8.0pt}%
\@x{}%
\@y{\@s{0}%
 Can node \ensuremath{i} sync from node \ensuremath{j}\.{?}
}%
\@xx{}%
\@x{ CanSyncFrom ( i ,\, j ) \.{\defeq}}%
\@x{\@s{16.4} \.{\land} Len ( Oplog [ i ] ) \.{<} Len ( Oplog [ j ] )}%
 \@x{\@s{16.4} \.{\land} LastTerm ( i ) \.{=} LogTerm ( j ,\, Len ( Oplog [ i
 ] ) )}%
\@pvspace{8.0pt}%
\@x{}%
\@y{\@s{0}%
 \ensuremath{Oplog} entries needed to replicate from \ensuremath{j} to
 \ensuremath{i
}}%
\@xx{}%
\@x{ ReplicateOplog ( i ,\, j ) \.{\defeq}}%
\@x{\@s{16.4} \.{\LET} len\_i\@s{0.42} \.{\defeq} Len ( Oplog [ i ] )}%
\@x{\@s{36.79} len\_j \.{\defeq} Len ( Oplog [ j ] )}%
 \@x{\@s{16.4} \.{\IN} {\IF} i \.{\in} Secondary \.{\land} len\_i \.{<}
 len\_j}%
 \@x{\@s{99.17} \.{\THEN} SubSeq ( Oplog [ j ] ,\, len\_i \.{+} 1 ,\, len\_j
 )}%
\@x{\@s{99.17} \.{\ELSE} {\langle} {\rangle}}%
\@pvspace{8.0pt}%
\@x{}%
\@y{\@s{0}%
 Can node \ensuremath{i} rollback its \ensuremath{log} based on
 \ensuremath{j}\mbox{'}s \ensuremath{log
}}%
\@xx{}%
 \@x{ CanRollback ( i ,\, j ) \.{\defeq} \.{\land} Len ( Oplog [ i ]
 )\@s{0.42} \.{>} 0}%
\@x{\@s{96.25} \.{\land} Len ( Oplog [ j ] ) \.{>} 0}%
\@x{\@s{96.25} \.{\land} CurrentTerm [ i ] \.{<} CurrentTerm [ j ]}%
\@x{\@s{96.25} \.{\land}}%
\@x{\@s{108.55} \.{\lor} Len ( Oplog [ i ] ) \.{>} Len ( Oplog [ j ] )}%
 \@x{\@s{108.55} \.{\lor} \.{\land} Len ( Oplog [ i ] ) \.{\leq} Len ( Oplog [
 j ] )}%
 \@x{\@s{119.66} \.{\land} CurrentTerm [ i ] \.{\neq} LogTerm ( j ,\, Len (
 Oplog [ i ] ) )}%
\@pvspace{8.0pt}%
\@x{}%
\@y{\@s{0}%
 Returns the highest common index between two divergent logs.
}%
\@xx{}%
\@x{}%
\@y{\@s{0}%
 If there is no common index between the logs, returns 0.
}%
\@xx{}%
\@x{ RollbackCommonPoint ( i ,\, j ) \.{\defeq}}%
 \@x{\@s{16.4} \.{\LET} commonIndices \.{\defeq} \{ k \.{\in} {\DOMAIN} Oplog
 [ i ] \.{:}}%
\@x{\@s{133.90} \.{\land} k \.{\leq} Len ( Oplog [ j ] )}%
 \@x{\@s{133.90} \.{\land} Oplog [ i ] [ k ] \.{=} Oplog [ j ] [ k ] \}
 \.{\IN}}%
 \@x{\@s{36.79} {\IF} commonIndices \.{=} \{ \} \.{\THEN} 0 \.{\ELSE} MaxVal (
 commonIndices )}%
\@pvspace{8.0pt}%
\@x{}%
\@y{\@s{0}%
 \ensuremath{Init} Part
}%
\@xx{}%
\@x{}\midbar\@xx{}%
 \@x{ InitPrimary \.{\defeq} Primary \.{=} \{ {\CHOOSE} s \.{\in} Server \.{:}
 {\TRUE} \}}%
 \@x{ InitSecondary \.{\defeq} Secondary \.{=} Server \.{\,\backslash\,}
 Primary}%
 \@x{ InitOplog \.{\defeq} Oplog \.{=} [ s\@s{1.47} \.{\in} Server \.{\mapsto}
 {\langle} {\rangle} ]}%
 \@x{ InitStore\@s{2.30} \.{\defeq} Store\@s{2.30} \.{=} [ n \.{\in} Server
 \.{\cup} Client\@s{4.1} \.{\mapsto} [ k \.{\in} Key \.{\mapsto} Nil ] ]}%
 \@x{ InitCt\@s{0.51} \.{\defeq} Ct\@s{0.51} \.{=} [ n\@s{4.46} \.{\in} Server
 \.{\cup} Client \.{\mapsto} [ p \.{\mapsto} 0 ,\, l \.{\mapsto} 0 ] ]}%
 \@x{ InitOt \.{\defeq} Ot \.{=} [ n\@s{4.46} \.{\in} Server \.{\cup} Client
 \.{\mapsto} [ p \.{\mapsto} 0 ,\, l \.{\mapsto} 0 ] ]}%
 \@x{ InitServerMsg \.{\defeq} ServerMsg \.{=} [ s \.{\in} Server \.{\mapsto}
 {\langle} {\rangle} ]}%
 \@x{ InitPt\@s{1.84} \.{\defeq} Pt\@s{1.84} \.{=} [ s\@s{1.47} \.{\in} Server
 \.{\mapsto} 1 ]}%
 \@x{ InitCp \.{\defeq} Cp \.{=} [ n \.{\in} Server \.{\cup} Client
 \.{\mapsto} [ p \.{\mapsto} 0 ,\, l \.{\mapsto} 0 ] ]}%
 \@x{ InitCalState \.{\defeq} CalState \.{=} [ s \.{\in} Server \.{\mapsto}
 CreateState ( Cardinality ( Server ) ,\, {\langle} {\rangle} ) ]}%
\@x{\@s{130.06}}%
\@y{\@s{0}%
 create initial \ensuremath{state(for} \ensuremath{calculate)
}}%
\@xx{}%
 \@x{ InitState \.{\defeq} State \.{=} [ s\@s{8.62} \.{\in} Server \.{\mapsto}
 [ s0 \.{\in} Server \.{\mapsto}}%
\@x{\@s{184.46} [ p \.{\mapsto} 0 ,\, l \.{\mapsto} 0 ] ] ]}%
 \@x{ InitCurrentTerm \.{\defeq} CurrentTerm \.{=} [ s \.{\in} Server
 \.{\mapsto} 0 ]}%
\@x{ InitReadyToServe \.{\defeq} ReadyToServe \.{=} 0}%
 \@x{ InitSyncSource \.{\defeq} SyncSource \.{=} [ s \.{\in} Server
 \.{\mapsto} Nil ]}%
\@pvspace{8.0pt}%
\@x{ Init \.{\defeq}}%
 \@x{\@s{16.4} \.{\land} InitPrimary \.{\land} InitSecondary \.{\land}
 InitOplog \.{\land} InitStore \.{\land} InitCt}%
 \@x{\@s{16.4} \.{\land} InitOt \.{\land} InitPt \.{\land} InitCp \.{\land}
 InitCalState}%
\@x{\@s{16.4} \.{\land} InitServerMsg}%
 \@x{\@s{16.4} \.{\land} InitState \.{\land} InitCurrentTerm \.{\land}
 InitReadyToServe}%
\@x{\@s{16.4} \.{\land} InitSyncSource}%
\@pvspace{8.0pt}%
\@x{}%
\@y{\@s{0}%
 Next \ensuremath{State} Actions
}%
\@xx{}%
\@x{}%
\@y{\@s{0}%
 Replication Protocol: possible actions
}%
\@xx{}%
\@x{}\midbar\@xx{}%
\@pvspace{8.0pt}%
\@x{ TurnOnReadyToServe \.{\defeq}}%
\@x{\@s{16.4} \.{\land} ReadyToServe \.{=} 0}%
\@x{\@s{16.4} \.{\land} \E\, s \.{\in} Primary \.{:}}%
 \@x{\@s{31.61} \.{\land} CurrentTerm \.{'} \.{=} [ CurrentTerm {\EXCEPT}
 {\bang} [ s ] \.{=} CurrentTerm [ s ] \.{+} 1 ]}%
\@x{\@s{31.61} \.{\land} ReadyToServe \.{'} \.{=} ReadyToServe \.{+} 1}%
 \@x{\@s{16.4} \.{\land} {\UNCHANGED} {\langle} Primary ,\,\@s{4.1} Secondary
 ,\, Oplog ,\, Store ,\, Ct ,\, Ot ,\,}%
\@x{\@s{89.98} ServerMsg ,\, Pt ,\, Cp ,\,}%
\@x{\@s{89.98} State ,\, CalState ,\, SyncSource {\rangle}}%
\@pvspace{8.0pt}%
\@x{ Stepdown \.{\defeq}}%
\@x{\@s{53.70} \.{\land} ReadyToServe \.{>} 0}%
\@x{\@s{53.70} \.{\land} \E\, s \.{\in} Primary \.{:}}%
 \@x{\@s{68.91} \.{\land} Primary \.{'} \.{=} Primary \.{\,\backslash\,} \{ s
 \}}%
\@x{\@s{68.91} \.{\land} Secondary \.{'} \.{=} Secondary \.{\cup} \{ s \}}%
 \@x{\@s{53.70} \.{\land} {\UNCHANGED} {\langle} Oplog ,\, Store ,\, Ct ,\, Ot
 ,\, ServerMsg ,\,}%
\@x{\@s{127.28} Pt ,\, Cp ,\, State ,\, CalState ,\, CurrentTerm ,\,}%
\@x{\@s{131.38} ReadyToServe ,\, SyncSource {\rangle}}%
\@pvspace{8.0pt}%
\@x{}%
\@y{\@s{0}%
 There are majority nodes agree to elect node \ensuremath{i} to become primary
}%
\@xx{}%
\@x{ ElectPrimary \.{\defeq}}%
\@x{\@s{16.4} \.{\land} ReadyToServe \.{>} 0}%
 \@x{\@s{16.4} \.{\land} \E\, i \.{\in} Server \.{:} \E\, majorNodes \.{\in}
 {\SUBSET} ( Server ) \.{:}}%
 \@x{\@s{31.61} \.{\land} \A\, j \.{\in} majorNodes \.{:} \.{\land} NotBehind
 ( i ,\, j )}%
\@x{\@s{128.46} \.{\land} CurrentTerm [ i ] \.{\geq} CurrentTerm [ j ]}%
\@x{\@s{31.61} \.{\land} IsMajority ( majorNodes )}%
\@x{\@s{27.51}}%
\@y{\@s{0}%
 voted nodes for \ensuremath{i} cannot be primary anymore
}%
\@xx{}%
 \@x{\@s{31.61} \.{\land} Primary \.{'} \.{=} \.{\LET} possiblePrimary
 \.{\defeq} Primary \.{\,\backslash\,} majorNodes}%
\@x{\@s{96.17} \.{\IN} possiblePrimary \.{\cup} \{ i \}}%
\@x{\@s{27.51}}%
\@y{\@s{0}%
 add voted nodes into secondaries
}%
\@xx{}%
 \@x{\@s{31.61} \.{\land} Secondary \.{'} \.{=} \.{\LET} possibleSecondary
 \.{\defeq} Secondary \.{\cup} majorNodes}%
\@x{\@s{103.57} \.{\IN} possibleSecondary \.{\,\backslash\,} \{ i \}}%
 \@x{\@s{31.61} \.{\land} CurrentTerm \.{'} \.{=} [ index \.{\in} Server
 \.{\mapsto} {\IF} index \.{\in} ( majorNodes \.{\cup} \{ i \} )}%
\@x{\@s{201.84} \.{\THEN} CurrentTerm [ i ] \.{+} 1}%
\@x{\@s{201.84} \.{\ELSE} CurrentTerm [ index ] ]}%
\@x{\@s{31.61}}%
\@y{\@s{0}%
 A primary node do not have any sync source
}%
\@xx{}%
 \@x{\@s{31.61} \.{\land} SyncSource \.{'} \.{=} [ SyncSource {\EXCEPT}
 {\bang} [ i ] \.{=} Nil ]}%
 \@x{\@s{31.61} \.{\land} {\UNCHANGED} {\langle} Oplog ,\, Store ,\, Ct ,\, Ot
 ,\, ServerMsg ,\, Pt ,\, Cp ,\, State ,\, CalState ,\,}%
\@x{\@s{105.19} ReadyToServe {\rangle}}%
\@pvspace{8.0pt}%
\@x{ AdvanceCp \.{\defeq}}%
\@x{\@s{16.4} \.{\land} ReadyToServe \.{>} 0}%
\@x{\@s{16.4} \.{\land} \E\, s\@s{7.67} \.{\in} Primary \.{:}}%
 \@x{\@s{31.61} Cp \.{'} \.{=} [ Cp {\EXCEPT} {\bang} [ s ] \.{=} CalState [ s
 ] [ Cardinality ( Server ) \.{\div} 2 \.{+} 1 ] ]}%
 \@x{\@s{16.4} \.{\land} {\UNCHANGED} {\langle} Primary ,\, Secondary ,\,
 Oplog ,\, Store ,\, Ct ,\, Ot ,\,}%
\@x{\@s{89.98} ServerMsg ,\,\@s{4.1} Pt ,\, CalState ,\,}%
 \@x{\@s{89.98} State ,\, CurrentTerm ,\, ReadyToServe ,\, SyncSource
 {\rangle}}%
\@pvspace{8.0pt}%
\@x{}%
\@y{%
 注意:heartbeat没有更新oplog,没有更新\ensuremath{Ot},也没有更新store状态
}%
\@xx{}%
\@x{ ServerTakeHeartbeat \.{\defeq}}%
\@x{\@s{16.4} \.{\land} ReadyToServe \.{>} 0}%
\@x{\@s{16.4} \.{\land} \E\, s \.{\in} Server \.{:}}%
\@x{\@s{31.61} \.{\land} Len ( ServerMsg [ s ] ) \.{\neq} 0\@s{4.1}}%
\@y{\@s{0}%
 message channel is not empty
}%
\@xx{}%
\@x{\@s{31.61} \.{\land} ServerMsg [ s ] [ 1 ] . type \.{=}\@w{heartbeat}}%
 \@x{\@s{31.61} \.{\land} Ct \.{'} \.{=} [ Ct {\EXCEPT} {\bang} [ s ] \.{=}
 HLCMax ( Ct [ s ] ,\, ServerMsg [ s ] [ 1 ] . ct ) ]}%
\@x{\@s{31.61} \.{\land} State \.{'} \.{=}}%
\@x{\@s{46.82} \.{\LET} SubHbState \.{\defeq} State [ s ]}%
 \@x{\@s{67.22} hb \.{\defeq} [ SubHbState {\EXCEPT} {\bang} [ ServerMsg [ s ]
 [ 1 ] . s ] \.{=}}%
\@x{\@s{99.23} ServerMsg [ s ] [ 1 ] . aot ]}%
\@x{\@s{46.82} \.{\IN} [ State {\EXCEPT} {\bang} [ s ] \.{=} hb ]}%
\@x{\@s{31.61} \.{\land} CalState \.{'} \.{=} \.{\LET} newcal \.{\defeq}}%
\@x{\@s{116.81} {\IF} s \.{\in} Primary}%
\@y{\@s{0}%
 primary node: update \ensuremath{CalState
}}%
\@xx{}%
\@x{\@s{133.87} \.{\THEN}\@s{4.1} [ CalState {\EXCEPT} {\bang} [ s ] \.{=}}%
 \@x{\@s{172.06} AdvanceState ( ServerMsg [ s ] [ 1 ] . aot ,\, State [ s ] [
 ServerMsg [ s ] [ 1 ] . s ] ,\, CalState [ s ] ) ]}%
\@x{\@s{116.81} \.{\ELSE} CalState}%
\@x{\@s{96.41} \.{\IN} newcal}%
\@x{\@s{31.61} \.{\land} Cp \.{'} \.{=} \.{\LET} newcp \.{\defeq}}%
\@x{\@s{71.75}}%
\@y{\@s{0}%
 primary node: compute new mcp
}%
\@xx{}%
 \@x{\@s{84.05} {\IF} s \.{\in} Primary \.{\THEN} CalState \.{'} [ s ] [
 Cardinality ( Server ) \.{\div} 2 \.{+} 1 ]}%
\@x{\@s{71.75}}%
\@y{\@s{0}%
 secondary node: update mcp
}%
\@xx{}%
 \@x{\@s{84.05} \.{\ELSE} {\IF} {\lnot} HLCLt ( ServerMsg [ s ] [ 1 ] . cp ,\,
 Cp [ s ] )}%
 \@x{\@s{127.52} \.{\land} {\lnot} HLCLt ( Ot [ s ] ,\, ServerMsg [ s ] [ 1 ]
 . cp )}%
\@x{\@s{100.45} \.{\THEN} ServerMsg [ s ] [ 1 ] . cp}%
\@x{\@s{92.15} \.{\ELSE} Cp [ s ]}%
\@x{\@s{71.75} \.{\IN} [ Cp {\EXCEPT} {\bang} [ s ] \.{=} newcp ]}%
 \@x{\@s{27.51} \.{\land} ServerMsg \.{'} \.{=} [ ServerMsg {\EXCEPT} {\bang}
 [ s ] \.{=} Tail ( @ ) ]}%
 \@x{\@s{27.51} \.{\land} CurrentTerm \.{'} \.{=} [ CurrentTerm {\EXCEPT}
 {\bang} [ s ] \.{=} Max ( CurrentTerm [ s ] ,\, ServerMsg [ s ] [ 1 ] . term
 ) ]}%
 \@x{\@s{16.4} \.{\land} {\UNCHANGED} {\langle} Primary ,\, Secondary ,\,
 Oplog ,\, Store ,\, Ot ,\, Pt ,\,}%
\@x{\@s{89.98} ReadyToServe ,\, SyncSource {\rangle}}%
\@pvspace{8.0pt}%
\@x{ ServerTakeUpdatePosition \.{\defeq}}%
\@x{\@s{16.4} \.{\land} ReadyToServe \.{>} 0}%
\@x{\@s{16.4} \.{\land} \E\, s \.{\in} Server \.{:}}%
\@x{\@s{31.61} \.{\land} Len ( ServerMsg [ s ] ) \.{\neq} 0\@s{4.1}}%
\@y{\@s{0}%
 message channel is not empty
}%
\@xx{}%
 \@x{\@s{31.61} \.{\land} ServerMsg [ s ] [ 1 ] . type
 \.{=}\@w{update\_position}}%
 \@x{\@s{31.61} \.{\land} Ct \.{'} \.{=} [ Ct {\EXCEPT} {\bang} [ s ] \.{=}
 HLCMax ( Ct [ s ] ,\, ServerMsg [ s ] [ 1 ] . ct ) ]}%
\@y{\@s{0}%
 update \ensuremath{ct} accordingly
}%
\@xx{}%
\@x{\@s{31.61} \.{\land} State \.{'} \.{=}}%
\@x{\@s{46.82} \.{\LET} SubHbState \.{\defeq} State [ s ]}%
 \@x{\@s{67.22} hb \.{\defeq} [ SubHbState {\EXCEPT} {\bang} [ ServerMsg [ s ]
 [ 1 ] . s ] \.{=}}%
\@x{\@s{99.23} ServerMsg [ s ] [ 1 ] . aot ]}%
\@x{\@s{46.82} \.{\IN} [ State {\EXCEPT} {\bang} [ s ] \.{=} hb ]}%
\@x{\@s{31.61} \.{\land} CalState \.{'} \.{=} \.{\LET} newcal \.{\defeq}}%
\@x{\@s{116.81} {\IF} s \.{\in} Primary}%
\@y{\@s{0}%
 primary node: update \ensuremath{CalState
}}%
\@xx{}%
\@x{\@s{133.87} \.{\THEN} [ CalState {\EXCEPT} {\bang} [ s ] \.{=}}%
 \@x{\@s{172.06} AdvanceState ( ServerMsg [ s ] [ 1 ] . aot ,\, State [ s ] [
 ServerMsg [ s ] [ 1 ] . s ] ,\, CalState [ s ] ) ]}%
\@x{\@s{116.81} \.{\ELSE} CalState}%
\@x{\@s{96.41} \.{\IN} newcal}%
\@x{\@s{31.61} \.{\land} Cp \.{'} \.{=} \.{\LET} newcp \.{\defeq}}%
\@x{\@s{71.75}}%
\@y{\@s{0}%
 primary node: compute new mcp
}%
\@xx{}%
 \@x{\@s{71.75} {\IF} s \.{\in} Primary \.{\THEN} CalState \.{'} [ s ] [
 Cardinality ( Server ) \.{\div} 2 \.{+} 1 ]}%
\@x{\@s{71.75}}%
\@y{\@s{0}%
 secondary node: update mcp
}%
\@xx{}%
 \@x{\@s{71.75} \.{\ELSE} {\IF} {\lnot} HLCLt ( ServerMsg [ s ] [ 1 ] . cp ,\,
 Cp [ s ] )}%
 \@x{\@s{103.06} \.{\land}\@s{1.04} {\lnot} HLCLt ( Ot [ s ] ,\, ServerMsg [ s
 ] [ 1 ] . cp )}%
\@x{\@s{71.75} \.{\THEN} ServerMsg [ s ] [ 1 ] . cp}%
\@x{\@s{71.75} \.{\ELSE} Cp [ s ]}%
\@x{\@s{71.75} \.{\IN} [ Cp {\EXCEPT} {\bang} [ s ] \.{=} newcp ]}%
 \@x{\@s{27.51} \.{\land} CurrentTerm \.{'} \.{=} [ CurrentTerm {\EXCEPT}
 {\bang} [ s ] \.{=} Max ( CurrentTerm [ s ] ,\, ServerMsg [ s ] [ 1 ] . term
 ) ]}%
 \@x{\@s{27.51} \.{\land} ServerMsg \.{'} \.{=} \.{\LET} newServerMsg
 \.{\defeq} [ ServerMsg {\EXCEPT} {\bang} [ s ] \.{=} Tail ( @ ) ]}%
 \@x{\@s{101.15} \.{\IN} ( \.{\LET}\@s{4.1} appendMsg \.{\defeq} [ type
 \.{\mapsto}\@w{update\_position} ,\, s \.{\mapsto} ServerMsg [ s ] [ 1 ] . s
 ,\, aot \.{\mapsto} ServerMsg [ s ] [ 1 ] . aot ,\,}%
 \@x{\@s{182.74} ct \.{\mapsto} ServerMsg [ s ] [ 1 ] . ct ,\, cp \.{\mapsto}
 ServerMsg [ s ] [ 1 ] . cp ,\, term \.{\mapsto} ServerMsg [ s ] [ 1 ] . term
 ]}%
 \@x{\@s{125.44} \.{\IN} ( \.{\LET} newMsg \.{\defeq} {\IF} s \.{\in} Primary
 \.{\lor} SyncSource [ s ] \.{=} Nil}%
\@x{\@s{240.68} \.{\THEN} newServerMsg}%
\@y{\@s{0}%
 If \ensuremath{s} is primary, accept the \ensuremath{msg}, else forward it
 to its sync source
}%
\@xx{}%
 \@x{\@s{224.42} \.{\ELSE} [ newServerMsg {\EXCEPT} {\bang} [ SyncSource [ s ]
 ] \.{=} Append ( newServerMsg [ SyncSource [ s ] ] ,\, appendMsg ) ]}%
\@x{\@s{149.73} \.{\IN} newMsg ) )}%
 \@x{\@s{16.4} \.{\land} {\UNCHANGED} {\langle} Primary ,\, Secondary ,\,
 Oplog ,\, Store ,\, Ot ,\,}%
\@x{\@s{89.98} Pt ,\, ReadyToServe ,\, SyncSource {\rangle}}%
\@pvspace{8.0pt}%
\@x{ NTPSync \.{\defeq}}%
\@y{\@s{0}%
 simplify \ensuremath{NTP} protocal
}%
\@xx{}%
\@x{\@s{16.4} \.{\land} ReadyToServe \.{>} 0}%
 \@x{\@s{16.4} \.{\land} Pt \.{'} \.{=} [ s \.{\in} Server \.{\mapsto} MaxPt
 ]}%
 \@x{\@s{16.4} \.{\land} {\UNCHANGED} {\langle} Primary ,\, Secondary ,\,
 Oplog ,\, Store ,\, Ct ,\, Ot ,\,}%
\@x{\@s{90.19} ServerMsg ,\, Cp ,\,}%
 \@x{\@s{90.19} CalState ,\, State ,\, CurrentTerm ,\, ReadyToServe ,\,
 SyncSource {\rangle}}%
\@pvspace{8.0pt}%
\@x{ AdvancePt \.{\defeq}}%
\@x{\@s{16.4} \.{\land} ReadyToServe \.{>} 0}%
\@x{\@s{16.4} \.{\land} \E\, s \.{\in} Server \.{:}}%
\@x{\@s{38.83} \.{\land} s \.{\in} Primary\@s{77.9}}%
\@y{\@s{0}%
 for simplicity
}%
\@xx{}%
\@x{\@s{38.83} \.{\land} Pt [ s ] \.{\leq} PtStop}%
 \@x{\@s{38.83} \.{\land} Pt \.{'} \.{=} [ Pt {\EXCEPT} {\bang} [ s ] \.{=} @
 \.{+} 1 ]}%
\@y{\@s{0}%
 advance physical time
}%
\@xx{}%
\@x{\@s{38.83} \.{\land} BroadcastHeartbeat ( s )\@s{42.59}}%
\@y{\@s{0}%
 broadcast heartbeat periodly
}%
\@xx{}%
 \@x{\@s{16.4} \.{\land} {\UNCHANGED} {\langle} Primary ,\, Secondary ,\,
 Oplog ,\, Store ,\, Ct ,\, Ot ,\, State ,\,}%
 \@x{\@s{89.98} Cp ,\, CalState ,\, CurrentTerm ,\, ReadyToServe ,\,
 SyncSource {\rangle}}%
\@pvspace{8.0pt}%
\@x{}%
\@y{\@s{0}%
 Replicate oplog from node \ensuremath{j} to node \ensuremath{i}, and update
 related structures accordingly
}%
\@xx{}%
\@x{\@s{4.1} Replicate \.{\defeq}}%
\@x{\@s{16.4} \.{\land} ReadyToServe \.{>} 0}%
\@x{\@s{16.4} \.{\land} \E\, i ,\, j \.{\in} Server \.{:}}%
\@x{\@s{31.61} \.{\land} CanSyncFrom ( i ,\, j )}%
\@y{\@s{0}%
 \ensuremath{i} can sync from \ensuremath{j} only need not to rollback
}%
\@xx{}%
\@x{\@s{31.61} \.{\land} i \.{\in} Secondary}%
 \@x{\@s{31.61} \.{\land} ReplicateOplog ( i ,\, j ) \.{\neq} {\langle}
 {\rangle}}%
 \@x{\@s{31.61} \.{\land} Oplog \.{'} \.{=} [ Oplog {\EXCEPT} {\bang} [ i ]
 \.{=} @ \.{\circ} ReplicateOplog ( i ,\, j ) ]}%
 \@x{\@s{31.61} \.{\land} Store \.{'}\@s{2.30} \.{=} [ Store {\EXCEPT} {\bang}
 [ i ]\@s{2.30} \.{=}\@s{4.1} Store [ j ] ]}%
 \@x{\@s{31.61} \.{\land} Ct \.{'}\@s{1.47} \.{=} [ Ct {\EXCEPT} {\bang} [ i
 ]\@s{1.47} \.{=} HLCMax ( Ct [ i ] ,\, Ct [ j ] ) ]\@s{2.94}}%
\@y{\@s{0}%
 update \ensuremath{Ct[i]
}}%
\@xx{}%
 \@x{\@s{31.61} \.{\land} Ot \.{'}\@s{0.96} \.{=} [ Ot {\EXCEPT} {\bang} [ i
 ]\@s{0.96} \.{=} HLCMax ( Ot [ i ] ,\, Ot [ j ] ) ]\@s{1.92}}%
\@y{\@s{0}%
 update \ensuremath{Ot[i]
}}%
\@xx{}%
 \@x{\@s{31.61} \.{\land} Cp \.{'} \.{=} [ Cp {\EXCEPT} {\bang} [ i ] \.{=}
 HLCMax ( Cp [ i ] ,\, Cp [ j ] ) ]}%
\@y{\@s{0}%
 update \ensuremath{Cp[i]
}}%
\@xx{}%
 \@x{\@s{31.61} \.{\land} CurrentTerm \.{'} \.{=} [ CurrentTerm {\EXCEPT}
 {\bang} [ i ] \.{=} Max ( CurrentTerm [ i ] ,\, CurrentTerm [ j ] ) ]}%
\@y{\@s{0}%
 update \ensuremath{CurrentTerm
}}%
\@xx{}%
\@x{\@s{31.61} \.{\land} State \.{'} \.{=}}%
\@x{\@s{65.45} \.{\LET} SubHbState \.{\defeq} State [ i ]}%
 \@x{\@s{85.85} hb \.{\defeq} [ SubHbState {\EXCEPT} {\bang} [ j ] \.{=} Ot [
 j ] ]}%
\@x{\@s{65.45} \.{\IN} [ State {\EXCEPT} {\bang} [ i ] \.{=} hb ]}%
\@y{\@s{0}%
 update \ensuremath{j}\mbox{'}s state \ensuremath{i} knows
}%
\@xx{}%
 \@x{\@s{31.61} \.{\land} \.{\LET} msg \.{\defeq} [ type
 \.{\mapsto}\@w{update\_position} ,\, s \.{\mapsto} i ,\, aot \.{\mapsto} Ot
 \.{'} [ i ] ,\, ct \.{\mapsto} Ct \.{'} [ i ] ,\, cp \.{\mapsto} Cp \.{'} [
 i ] ,\, term \.{\mapsto} CurrentTerm \.{'} [ i ] ]}%
 \@x{\@s{42.72} \.{\IN} ServerMsg \.{'} \.{=} [ ServerMsg {\EXCEPT} {\bang} [
 j ] \.{=} Append ( ServerMsg [ j ] ,\, msg ) ]}%
 \@x{\@s{31.61} \.{\land} SyncSource \.{'} \.{=} [ SyncSource {\EXCEPT}
 {\bang} [ i ] \.{=} j ]}%
 \@x{\@s{31.61} \.{\land} CalState \.{'} \.{=} [ CalState {\EXCEPT} {\bang} [
 i ] \.{=} CalState [ j ] ]}%
 \@x{\@s{31.61} \.{\land} {\UNCHANGED} {\langle} Primary ,\, Secondary ,\, Pt
 ,\, ReadyToServe {\rangle}}%
\@pvspace{8.0pt}%
\@x{}%
\@y{\@s{0}%
 Rollback \ensuremath{i}\mbox{'}s oplog and recover it to
 \ensuremath{j}\mbox{'}s state
}%
\@xx{}%
\@x{}%
\@y{\@s{0}%
 Recover to \ensuremath{j}\mbox{'}s state immediately to prevent internal
 client request
}%
\@xx{}%
\@x{ RollbackAndRecover \.{\defeq}}%
\@x{\@s{16.4} \.{\land} ReadyToServe \.{>} 0}%
\@x{\@s{16.4} \.{\land} \E\, i ,\, j \.{\in} Server \.{:}}%
\@x{\@s{31.61} \.{\land} i \.{\in} Secondary}%
\@x{\@s{31.61} \.{\land} CanRollback ( i ,\, j )}%
 \@x{\@s{31.61} \.{\land} \.{\LET} cmp \.{\defeq} RollbackCommonPoint ( i ,\,
 j )\@s{4.1} \.{\IN}}%
 \@x{\@s{42.72} \.{\LET} commonLog \.{\defeq} SubSeq ( Oplog [ i ] ,\, 1 ,\,
 cmp )}%
 \@x{\@s{63.12} appendLog\@s{6.13} \.{\defeq} SubSeq ( Oplog [ j ] ,\, cmp
 \.{+} 1 ,\, Len ( Oplog [ j ] ) )}%
 \@x{\@s{42.72} \.{\IN} Oplog \.{'} \.{=} [ Oplog {\EXCEPT} {\bang} [ i ]
 \.{=} commonLog \.{\circ} appendLog ]}%
 \@x{\@s{31.61} \.{\land} CurrentTerm \.{'} \.{=} [ CurrentTerm {\EXCEPT}
 {\bang} [ i ] \.{=} Max ( CurrentTerm [ i ] ,\, CurrentTerm [ j ] ) ]}%
\@y{\@s{0}%
 update \ensuremath{CurrentTerm
}}%
\@xx{}%
 \@x{\@s{31.61} \.{\land} Store \.{'} \.{=} [ Store {\EXCEPT} {\bang} [ i ]
 \.{=}\@s{4.1} Store [ j ] ]}%
 \@x{\@s{31.61} \.{\land} Ct \.{'}\@s{1.47} \.{=} [ Ct {\EXCEPT} {\bang} [ i
 ]\@s{1.47} \.{=} HLCMax ( Ct [ i ] ,\, Ct [ j ] ) ]\@s{2.94}}%
\@y{\@s{0}%
 update \ensuremath{Ct[i]
}}%
\@xx{}%
 \@x{\@s{31.61} \.{\land} Ot \.{'}\@s{0.96} \.{=} [ Ot {\EXCEPT} {\bang} [ i
 ]\@s{0.96} \.{=} HLCMax ( Ot [ i ] ,\, Ot [ j ] ) ]\@s{1.92}}%
\@y{\@s{0}%
 update \ensuremath{Ot[i]
}}%
\@xx{}%
 \@x{\@s{31.61} \.{\land} Cp \.{'} \.{=} [ Cp {\EXCEPT} {\bang} [ i ] \.{=}
 HLCMax ( Cp [ i ] ,\, Cp [ j ] ) ]}%
\@y{\@s{0}%
 update \ensuremath{Cp[i]
}}%
\@xx{}%
 \@x{\@s{31.61} \.{\land} State \.{'} \.{=} \.{\LET} SubHbState \.{\defeq}
 State [ i ]}%
 \@x{\@s{101.98} hb \.{\defeq} [ SubHbState {\EXCEPT} {\bang} [ j ] \.{=} Ot [
 j ] ]}%
\@x{\@s{81.58} \.{\IN} [ State {\EXCEPT} {\bang} [ i ] \.{=} hb ]}%
\@y{\@s{0}%
 update \ensuremath{j}\mbox{'}s state \ensuremath{i} knows
}%
\@xx{}%
 \@x{\@s{31.61} \.{\land} \.{\LET} msg \.{\defeq} [ type
 \.{\mapsto}\@w{update\_position} ,\, s \.{\mapsto} i ,\, aot \.{\mapsto} Ot
 \.{'} [ i ] ,\, ct \.{\mapsto} Ct \.{'} [ i ] ,\, cp \.{\mapsto} Cp \.{'} [
 i ] ]}%
 \@x{\@s{42.72} \.{\IN} ServerMsg \.{'} \.{=} [ ServerMsg {\EXCEPT} {\bang} [
 j ] \.{=} Append ( ServerMsg [ j ] ,\, msg ) ]}%
 \@x{\@s{31.61} \.{\land} SyncSource \.{'} \.{=} [ SyncSource {\EXCEPT}
 {\bang} [ i ] \.{=} j ]}%
 \@x{\@s{31.61} \.{\land} {\UNCHANGED} {\langle} Primary ,\, Secondary ,\, Pt
 ,\, CalState ,\,}%
\@x{\@s{75.52} ReadyToServe {\rangle}}%
\@pvspace{8.0pt}%
\@x{ ClientRequest \.{\defeq}}%
\@x{\@s{16.4} \.{\land} ReadyToServe \.{>} 0}%
 \@x{\@s{16.4} \.{\land} \E\, s \.{\in} Server ,\, k \.{\in} Key ,\, v \.{\in}
 Value \.{:}}%
\@x{\@s{31.61} \.{\land} s \.{\in} Primary}%
\@x{\@s{31.61} \.{\land} Tick ( s )}%
 \@x{\@s{31.61} \.{\land} Ot \.{'} \.{=} [ Ot {\EXCEPT} {\bang} [ s ] \.{=} Ct
 \.{'} [ s ] ]}%
 \@x{\@s{31.61} \.{\land} Store \.{'}\@s{2.30} \.{=} [ Store {\EXCEPT} {\bang}
 [ s ] [ k ] \.{=} v ]}%
 \@x{\@s{31.61} \.{\land} Oplog \.{'} \.{=} \.{\LET} entry \.{\defeq} [ k
 \.{\mapsto} k ,\, v \.{\mapsto} v ,\, ot \.{\mapsto} Ot \.{'} [ s ] ,\, term
 \.{\mapsto} CurrentTerm [ s ] ]}%
\@x{\@s{108.77} newLog \.{\defeq} Append ( Oplog [ s ] ,\, entry )}%
 \@x{\@s{84.27} \.{\IN}\@s{4.09} [ Oplog {\EXCEPT} {\bang} [ s ] \.{=} newLog
 ]}%
 \@x{\@s{31.61} \.{\land} State \.{'}\@s{2.68} \.{=} \.{\LET} SubHbState
 \.{\defeq} State [ s ]}%
 \@x{\@s{104.67} hb \.{\defeq} [ SubHbState {\EXCEPT} {\bang} [ s ] \.{=} Ot
 \.{'} [ s ] ]}%
\@x{\@s{84.27} \.{\IN} [ State {\EXCEPT} {\bang} [ s ] \.{=} hb ]}%
 \@x{\@s{31.61} \.{\land} CalState \.{'} \.{=} [ CalState {\EXCEPT} {\bang} [
 s ] \.{=}}%
 \@x{\@s{161.34} AdvanceState ( Ot \.{'} [ s ] ,\, Ot [ s ] ,\, CalState [ s ]
 ) ]}%
 \@x{\@s{31.61} \.{\land} {\UNCHANGED} {\langle} Primary ,\,\@s{4.1} Secondary
 ,\, ServerMsg ,\,}%
\@x{\@s{105.19} Pt ,\, Cp ,\,}%
\@x{\@s{105.19} CurrentTerm ,\, ReadyToServe ,\, SyncSource {\rangle}}%
\@pvspace{8.0pt}%
\@x{}%
\@y{\@s{0}%
 Next state for all configurations
}%
\@xx{}%
\@x{ Next \.{\defeq} \.{\lor} Replicate}%
\@x{\@s{39.83} \.{\lor} AdvancePt}%
\@x{\@s{39.83} \.{\lor} ServerTakeHeartbeat}%
\@x{\@s{39.83} \.{\lor} ServerTakeUpdatePosition}%
\@x{\@s{39.83} \.{\lor} Stepdown}%
\@x{\@s{39.83} \.{\lor} RollbackAndRecover}%
\@x{\@s{39.83} \.{\lor} TurnOnReadyToServe}%
\@x{\@s{39.83} \.{\lor} ElectPrimary}%
\@x{\@s{39.83} \.{\lor} ClientRequest}%
\@x{\@s{39.83} \.{\lor} NTPSync}%
\@pvspace{8.0pt}%
\@x{ Spec\@s{1.46} \.{\defeq} Init \.{\land} {\Box} [ Next ]_{ vars}}%
\@pvspace{8.0pt}%
\@x{}\midbar\@xx{}%
\@x{}%
\@y{%
 Properties to check\.{?}
}%
\@xx{}%
\@pvspace{8.0pt}%
\@x{ IsLogPrefix ( i ,\, j ) \.{\defeq}}%
\@x{\@s{16.4} \.{\land} Len ( Oplog [ i ] ) \.{\leq} Len ( Oplog [ j ] )}%
 \@x{\@s{16.4} \.{\land} Oplog [ i ] \.{=} SubSeq ( Oplog [ j ] ,\, 1 ,\, Len
 ( Oplog [ i ] ) )}%
\@pvspace{8.0pt}%
\@x{}%
\@y{\@s{0}%
 If two logs have the same last \ensuremath{log} entry term, then one is a
 prefix of the other (from Will)
}%
\@xx{}%
\@x{ LastTermsEquivalentImplyPrefixes \.{\defeq}}%
\@x{\@s{16.4} \A\, i ,\, j \.{\in} Server \.{:}}%
 \@x{\@s{27.70} LogTerm ( i ,\, Len ( Oplog [ i ] ) ) \.{=} LogTerm ( j ,\,
 Len ( Oplog [ j ] ) ) \.{\implies}}%
\@x{\@s{27.70} IsLogPrefix ( i ,\, j ) \.{\lor} IsLogPrefix ( j ,\, i )}%
\@x{}%
\@y{\@s{0}%
 Check whether terms are incremented monotoniclly (from Will
}%
\@xx{}%
\@x{\@s{4.1} TermsMonotonic \.{\defeq}}%
 \@x{\@s{16.4} {\Box} [ \A\, s \.{\in} Server \.{:} CurrentTerm \.{'} [ s ]
 \.{\geq} CurrentTerm [ s ] ]_{ vars}}%
\@pvspace{8.0pt}%
\@x{}%
\@y{\@s{0}%
 Check the \ensuremath{log} in \ensuremath{Primary} node is append only (from
 Will
}%
\@xx{}%
\@x{ PrimaryAppendOnly \.{\defeq}}%
 \@x{\@s{16.4} {\Box} [ \A\, s \.{\in} Server \.{:} s \.{\in} Primary
 \.{\implies} Len ( Oplog \.{'} [ s ] ) \.{\geq} Len ( Oplog [ s ] ) ]_{
 vars}}%
\@pvspace{8.0pt}%
\@x{}%
\@y{\@s{0}%
 Never rollback oplog before common point (from Will \& Raft \ensuremath{Mongo
}}%
\@xx{}%
\@x{ NeverRollbackCommonPoint \.{\defeq}}%
 \@x{\@s{16.4} \E\, i ,\, j \.{\in} Server \.{:} CanRollback ( i ,\, j )
 \.{\implies}}%
 \@x{\@s{27.70} \.{\LET} commonPoint \.{\defeq} RollbackCommonPoint ( i ,\, j
 )}%
\@x{\@s{48.10} lastOplog \.{\defeq} Oplog [ i ] [ commonPoint ]}%
\@x{\@s{27.70} \.{\IN} HLCLt ( Cp [ i ] ,\, lastOplog . ot )}%
\@pvspace{8.0pt}%
\@x{}%
\@y{\@s{0}%
 Eventually \ensuremath{log} correctness (from Will
}%
\@xx{}%
 \@x{ EventuallyLogsConverge\@s{6.55} \.{\defeq} {\Diamond} {\Box} [ \A\, s
 ,\, t \.{\in} Server \.{:} s \.{\neq} t \.{\implies} Oplog [ s ] \.{=} Oplog
 [ t ] ]_{ vars}}%
 \@x{ EventuallyLogsNonEmpty \.{\defeq} {\Diamond} ( \E\, s \.{\in} Server
 \.{:} Len ( Oplog [ s ] ) \.{>} 0 )}%
\@pvspace{8.0pt}%
\@x{}%
\@y{\@s{0}%
 (from \ensuremath{RaftMongo
}}%
\@xx{}%
\@x{ TwoPrimariesInSameTerm \.{\defeq}}%
\@x{\@s{16.4} \E\, i ,\, j \.{\in} Server \.{:}}%
\@x{\@s{27.70} \.{\land} i \.{\neq} j}%
\@x{\@s{27.70} \.{\land} CurrentTerm [ i ] \.{=} CurrentTerm [ j ]}%
\@x{\@s{27.70} \.{\land} i\@s{0.42} \.{\in} Primary}%
\@x{\@s{27.70} \.{\land} j \.{\in} Primary}%
\@pvspace{8.0pt}%
\@x{ NoTwoPrimariesInSameTerm \.{\defeq} {\lnot} TwoPrimariesInSameTerm}%
\@pvspace{8.0pt}%
\@x{}%
\@y{\@s{0}%
 Check if there is any cycle of sync source path (from \ensuremath{RaftMongo}
 \ensuremath{Sync
}}%
\@xx{}%
\@x{ SyncSourceCycleTwoNode \.{\defeq}}%
\@x{\@s{16.4} \E\, s ,\, t \.{\in} Server \.{:}}%
\@x{\@s{28.53} \.{\land} s \.{\neq} t}%
\@x{\@s{28.53} \.{\land} SyncSource [ s ] \.{=} t}%
\@x{\@s{28.53} \.{\land} SyncSource [ t ]\@s{0.63} \.{=} s}%
\@pvspace{8.0pt}%
\@x{ BoundedSeq ( s ,\, n ) \.{\defeq} [ 1 \.{\dotdot} n \.{\rightarrow} s ]}%
\@pvspace{8.0pt}%
\@x{ SyncSourcePaths \.{\defeq}}%
 \@x{\@s{16.4} \{ p \.{\in} BoundedSeq ( Server ,\, Cardinality ( Server ) )
 \.{:}}%
 \@x{\@s{27.14} \A\, i \.{\in} 1 \.{\dotdot} ( Len ( p ) \.{-} 1 ) \.{:}
 SyncSource [ p [ i ] ] \.{=} p [ i \.{+} 1 ] \}}%
\@pvspace{8.0pt}%
\@x{ SyncSourcePath ( i ,\, j ) \.{\defeq}}%
\@x{\@s{16.4} \E\, p \.{\in} SyncSourcePaths \.{:}}%
\@x{\@s{27.72} \.{\land} Len ( p ) \.{>} 1}%
\@x{\@s{27.72} \.{\land} p [ 1 ] \.{=} i}%
\@x{\@s{27.72} \.{\land} p [ Len ( p ) ] \.{=} j}%
\@pvspace{8.0pt}%
\@x{ SyncSourceCycle \.{\defeq}}%
\@x{\@s{16.4} \E\, s \.{\in} Server \.{:} SyncSourcePath ( s ,\, s )}%
\@pvspace{8.0pt}%
 \@x{ NonTrivialSyncCycle \.{\defeq} SyncSourceCycle \.{\land} {\lnot}
 SyncSourceCycleTwoNode}%
\@x{ NoNonTrivialSyncCycle \.{\defeq} {\lnot} NonTrivialSyncCycle}%
\@pvspace{16.0pt}%
\@x{}\bottombar\@xx{}%
\setboolean{shading}{false}
\begin{lcom}{0}%
\begin{cpar}{0}{F}{F}{0}{0}{}%
\ensuremath{\.{\,\backslash\,}\.{*}} Modification \ensuremath{History
}%
\end{cpar}%
\begin{cpar}{0}{F}{F}{0}{0}{}%
 \ensuremath{\.{\,\backslash\,}\.{*}} Last modified \ensuremath{Fri}
 \ensuremath{Apr} 22 10:10:51 \ensuremath{CST} 2022 by \ensuremath{dh
}%
\end{cpar}%
\begin{cpar}{0}{F}{F}{0}{0}{}%
 \ensuremath{\.{\,\backslash\,}\.{*}} Created \ensuremath{Mon}
 \ensuremath{Apr} 18 11:38:53 \ensuremath{CST} 2022 by \ensuremath{dh
}%
\end{cpar}%
\end{lcom}%
\end{document}
